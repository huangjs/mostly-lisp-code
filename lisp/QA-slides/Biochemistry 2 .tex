\documentclass[11pt]{beamer}
\beamersetaveragebackground{green!80!gray}
\usepackage[utf8]{inputenc}

\title{First Aid 2010 (Pharm)}
\author{huangjs}

\begin{document}


%% page: 1
\begin{frame}
 \frametitle{Biochemistry 2}
B1
\end{frame}

%% page: 2
\begin{frame}
 \frametitle{}
\begin{itemize}
\item Thiamine, TPP
\end{itemize}
\end{frame}

%% page: 3
\begin{frame}
 \frametitle{Biochemistry 2}
B2
\end{frame}

%% page: 4
\begin{frame}
 \frametitle{}
\begin{itemize}
\item riboflavin, FAD, FMN
\end{itemize}
\end{frame}

%% page: 5
\begin{frame}
 \frametitle{Biochemistry 2}
B3
\end{frame}

%% page: 6
\begin{frame}
 \frametitle{}
\begin{itemize}
\item Niacin, NAD+
\end{itemize}
\end{frame}

%% page: 7
\begin{frame}
 \frametitle{Biochemistry 2}
B5
\end{frame}

%% page: 8
\begin{frame}
 \frametitle{}
\begin{itemize}
\item pantothenic acid, CoA 
\end{itemize}
\end{frame}

%% page: 9
\begin{frame}
 \frametitle{Biochemistry 2}
B6
\end{frame}

%% page: 10
\begin{frame}
 \frametitle{}
\begin{itemize}
\item pyridoxine, PLP
\end{itemize}
\end{frame}

%% page: 11
\begin{frame}
 \frametitle{Biochemistry 2}
B-complex deficiencies 
\end{frame}

%% page: 12
\begin{frame}
 \frametitle{}
\begin{itemize}
\item often result in dermatitis, glossitis, and diarrhea. \\ All water soluble vitamins wash out easily from body except B12 and folate (stored in liver).
\end{itemize}
\end{frame}

%% page: 13
\begin{frame}
 \frametitle{Biochemistry 2}
B12 
\end{frame}

%% page: 14
\begin{frame}
 \frametitle{}
\begin{itemize}
\item cobalamin
\end{itemize}
\end{frame}

%% page: 15
\begin{frame}
 \frametitle{Biochemistry 2}
C
\end{frame}

%% page: 16
\begin{frame}
 \frametitle{}
\begin{itemize}
\item ascorbic acid 
\end{itemize}
\end{frame}

%% page: 17
\begin{frame}
 \frametitle{Biochemistry 2}
Vitamin A (retinol) function 
\end{frame}

%% page: 18
\begin{frame}
 \frametitle{}
\begin{itemize}
\item antioxidant, constituent of visual pigments ( retinal), essential for normal differentiation of epithelial cells into specialized tissue ( pancreatic cells, mucus-secreting cells) 
\end{itemize}
\end{frame}

%% page: 19
\begin{frame}
 \frametitle{Biochemistry 2}
vitamin A (retinol) deficiency  
\end{frame}

%% page: 20
\begin{frame}
 \frametitle{}
\begin{itemize}
\item night blindeness, dry skin
\end{itemize}
\end{frame}

%% page: 21
\begin{frame}
 \frametitle{Biochemistry 2}
vitamin A (retinol) excess  
\end{frame}

%% page: 22
\begin{frame}
 \frametitle{}
\begin{itemize}
\item Arthralgias, fatigue, headaches, skin changes, sore \\ throat, alopecia. \\ Teratogenic ( cleft palate, cardiac abnormalities), pregnancy test must be done before isotretinoin is prescribed for severe acne. 
\end{itemize}
\end{frame}

%% page: 23
\begin{frame}
 \frametitle{Biochemistry 2}
Retinol
\end{frame}

%% page: 24
\begin{frame}
 \frametitle{}
\begin{itemize}
\item is vitamin A, so think Retin-A (used topically for wrinkles and acne). \\ Found in leafy vegetables.
\end{itemize}
\end{frame}

%% page: 25
\begin{frame}
 \frametitle{Biochemistry 2}
Wernicke korsakoff 
\end{frame}

%% page: 26
\begin{frame}
 \frametitle{}
\begin{itemize}
\item confusion, ophthalmoplegis, ataxia and memory loss, confabulation, personality change. 
\end{itemize}
\end{frame}

%% page: 27
\begin{frame}
 \frametitle{Biochemistry 2}
Vitamin B1 ( Thiamine) deficiency 
\end{frame}

%% page: 28
\begin{frame}
 \frametitle{}
\begin{itemize}
\item Impaired glucose breakdown---ATP depletion. Highly aerobic tissues ( brain and heart ) are affected first. Wernicke-Korsakoff syndrome and beriberi. Seen in malnutrition as well as alcoholism ( secondary to malnutrition and malabsorption.) 
\end{itemize}
\end{frame}

%% page: 29
\begin{frame}
 \frametitle{Biochemistry 2}
cheilosis 
\end{frame}

%% page: 30
\begin{frame}
 \frametitle{}
\begin{itemize}
\item inflammation of lips, scaling and fissues at the corners of the mouth), corneal vascularization.  \\ vitamin B2 ( riboflavin deficiency) 
\end{itemize}
\end{frame}

%% page: 31
\begin{frame}
 \frametitle{Biochemistry 2}
Vitamin B3 excess 
\end{frame}

%% page: 32
\begin{frame}
 \frametitle{}
\begin{itemize}
\item facial flushing ( due to pharmacologic doses for treatment of hyperlipidemia). Vitamin B3 in corn not absorbable unless treated. Excess untreated corn in diet can lead to pellagra ( deficiency) 
\end{itemize}
\end{frame}

%% page: 33
\begin{frame}
 \frametitle{Biochemistry 2}
vitamin B6 ( pyridoxine) deficiency 
\end{frame}

%% page: 34
\begin{frame}
 \frametitle{}
\begin{itemize}
\item sideroblastic anemias, convulsion.....
\end{itemize}
\end{frame}

%% page: 35
\begin{frame}
 \frametitle{Biochemistry 2}
Vitamin B6 function 
\end{frame}

%% page: 36
\begin{frame}
 \frametitle{}
\begin{itemize}
\item cystathionine synthesis, heme syn, glycogen phosphorylase, decarboxylation reactions, ALT and AST transamination 
\end{itemize}
\end{frame}

%% page: 37
\begin{frame}
 \frametitle{Biochemistry 2}
Vitamin B12 deficiency caused by
\end{frame}

%% page: 38
\begin{frame}
 \frametitle{}
\begin{itemize}
\item malabsorption (sprue, enteritis, Diphyllobothrium latum), lack of intrinsic factor (pernicious anemia, gastric bypass surgery), or
absence of terminal ileum
(Crohn’s disease). Use Schilling test to detect
the etiology of the deficiency.
\end{itemize}
\end{frame}

%% page: 39
\begin{frame}
 \frametitle{Biochemistry 2}
Abnormal myelin is seen in B12 deficiency 
\end{frame}

%% page: 40
\begin{frame}
 \frametitle{}
\begin{itemize}
\item due to ${\downarrow}$ methionine or ${\uparrow}$ methylmalonic acid (from metabolism of accumulated methylmalonyl-CoA).
\end{itemize}
\end{frame}

%% page: 41
\begin{frame}
 \frametitle{Biochemistry 2}
Drugs cause folic acid deficiency 
\end{frame}

%% page: 42
\begin{frame}
 \frametitle{}
\begin{itemize}
\item phenytoin, sulfonamides, MTX, seen in alcoholism and pregnancy. 
\end{itemize}
\end{frame}

%% page: 43
\begin{frame}
 \frametitle{Biochemistry 2}
SAM 
\end{frame}

%% page: 44
\begin{frame}
 \frametitle{}
\begin{itemize}
\item ATP + methionine=== SAM. Regeneration of methionine is dependent on vitamin B12 and folate.  \\ Sam transfers methyl units.  \\ Sam required for the conversion NE to epinephrine. 
\end{itemize}
\end{frame}

%% page: 45
\begin{frame}
 \frametitle{Biochemistry 2}
Biotin 
\end{frame}

%% page: 46
\begin{frame}
 \frametitle{}
\begin{itemize}
\item cofactor for carboxylation enzymes ( which add a 1-carbon group)
\end{itemize}
\end{frame}

%% page: 47
\begin{frame}
 \frametitle{Biochemistry 2}
D2
\end{frame}

%% page: 48
\begin{frame}
 \frametitle{}
\begin{itemize}
\item ergocalciferol, consumed in milk.
\end{itemize}
\end{frame}

%% page: 49
\begin{frame}
 \frametitle{Biochemistry 2}
D3 
\end{frame}

%% page: 50
\begin{frame}
 \frametitle{}
\begin{itemize}
\item cholecalciferol, formed in sun-exposed skin
\end{itemize}
\end{frame}

%% page: 51
\begin{frame}
 \frametitle{Biochemistry 2}
storage form of vitamin D 
\end{frame}

%% page: 52
\begin{frame}
 \frametitle{}
\begin{itemize}
\item 25-OH D3 
\end{itemize}
\end{frame}

%% page: 53
\begin{frame}
 \frametitle{Biochemistry 2}
active form of vitamin D 
\end{frame}

%% page: 54
\begin{frame}
 \frametitle{}
\begin{itemize}
\item 1,25 (OH)2 D3 (calcitriol)
\end{itemize}
\end{frame}

%% page: 55
\begin{frame}
 \frametitle{Biochemistry 2}
Excess of vitamin D 
\end{frame}

%% page: 56
\begin{frame}
 \frametitle{}
\begin{itemize}
\item Hypercalcemia, loss of appetite, stupor. Seen in sarcoidosis, a disease where the epithelioid macrophages convert vitamin D into its active form.
\end{itemize}
\end{frame}

%% page: 57
\begin{frame}
 \frametitle{Biochemistry 2}
Vitamin E function 
\end{frame}

%% page: 58
\begin{frame}
 \frametitle{}
\begin{itemize}
\item Antioxidant ( protect erythrocytes and membrances from free-radical damage) 
\end{itemize}
\end{frame}

%% page: 59
\begin{frame}
 \frametitle{Biochemistry 2}
Vitamin K Function
\end{frame}

%% page: 60
\begin{frame}
 \frametitle{}
\begin{itemize}
\item Catalyzes ${\gamma}$-carboxylation of glutamic acid residues on various proteins concerned with blood clotting. Synthesized by intestinal flora. Therefore, vitamin K deficiency can occur after the prolonged use of broad-spectrum antibiotics.
\end{itemize}
\end{frame}

%% page: 61
\begin{frame}
 \frametitle{Biochemistry 2}
vitamin K deficiency 
\end{frame}

%% page: 62
\begin{frame}
 \frametitle{}
\begin{itemize}
\item Neonatal hemorrhage with ${\uparrow}$ PT and ${\uparrow}$ aPTT but normal bleeding time, because neonates have sterile intestines and are unable to synthesize vitamin K.
\end{itemize}
\end{frame}

%% page: 63
\begin{frame}
 \frametitle{Biochemistry 2}
zinc function 
\end{frame}

%% page: 64
\begin{frame}
 \frametitle{}
\begin{itemize}
\item essential for the activity of 100+ enzymes. Important in the formation of zinc fingers ( transcription factor motif) 
\end{itemize}
\end{frame}

%% page: 65
\begin{frame}
 \frametitle{Biochemistry 2}
ethanol metabolism 
\end{frame}

%% page: 66
\begin{frame}
 \frametitle{}
\begin{itemize}
\item ethanol ====acetaldehyde ( located in cytosol) ====acetate ( located in mitochondria)  \\ NAD+ is the limiting reagent, alcohol dehydrogenase operates via zero-order kinetics  \\ ethanol metabolism use up NAD+ 
\end{itemize}
\end{frame}

%% page: 67
\begin{frame}
 \frametitle{Biochemistry 2}
marasmus 
\end{frame}

%% page: 68
\begin{frame}
 \frametitle{}
\begin{itemize}
\item energy malnutrition resulting in tissue and muscle wasting, loss of subcutaneous fat, and variable edema. 
\end{itemize}
\end{frame}

%% page: 69
\begin{frame}
 \frametitle{Biochemistry 2}
kwashiorkor fatty liver 
\end{frame}

%% page: 70
\begin{frame}
 \frametitle{}
\begin{itemize}
\item fatty change due to decrease apolipoprotein syn 
\end{itemize}
\end{frame}

%% page: 71
\begin{frame}
 \frametitle{Biochemistry 2}
Hug takes two metabolism sites
\end{frame}

%% page: 72
\begin{frame}
 \frametitle{}
\begin{itemize}
\item Heme syn, Urea cycle, Gluconeogenesis 
\end{itemize}
\end{frame}

%% page: 73
\begin{frame}
 \frametitle{Biochemistry 2}
rate derterminging enzyme of glycolysis 
\end{frame}

%% page: 74
\begin{frame}
 \frametitle{}
\begin{itemize}
\item Phosphofructokinase-1 ( PFK-1) 
\end{itemize}
\end{frame}

%% page: 75
\begin{frame}
 \frametitle{Biochemistry 2}
De novo pyrimidine syn
\end{frame}

%% page: 76
\begin{frame}
 \frametitle{}
\begin{itemize}
\item carbamoyl phosphate synthetase II 
\end{itemize}
\end{frame}

%% page: 77
\begin{frame}
 \frametitle{Biochemistry 2}
gluconeogenesis 
\end{frame}

%% page: 78
\begin{frame}
 \frametitle{}
\begin{itemize}
\item fructose-1,6-bisphosphatase
\end{itemize}
\end{frame}

%% page: 79
\begin{frame}
 \frametitle{Biochemistry 2}
glucokinase 
\end{frame}

%% page: 80
\begin{frame}
 \frametitle{}
\begin{itemize}
\item an enzyme that catalyzes the phosphorylation of glucose using a molecule of ATP
\end{itemize}
\end{frame}

%% page: 81
\begin{frame}
 \frametitle{Biochemistry 2}
Kinase 
\end{frame}

%% page: 82
\begin{frame}
 \frametitle{}
\begin{itemize}
\item Use ATP to add high-energy phosphate group onto substrate ( e.g. phosphofructokinase) 
\end{itemize}
\end{frame}

%% page: 83
\begin{frame}
 \frametitle{Biochemistry 2}
phosphorylase 
\end{frame}

%% page: 84
\begin{frame}
 \frametitle{}
\begin{itemize}
\item adds inorganic phosphate onto substrate without using ATP ( glycogen phosphorylase) 
\end{itemize}
\end{frame}

%% page: 85
\begin{frame}
 \frametitle{Biochemistry 2}
phosphatase 
\end{frame}

%% page: 86
\begin{frame}
 \frametitle{}
\begin{itemize}
\item removes phosphate group from substrate ( e. g. fructose-1,6-biphosphatase) 
\end{itemize}
\end{frame}

%% page: 87
\begin{frame}
 \frametitle{Biochemistry 2}
dehydrogenase 
\end{frame}

%% page: 88
\begin{frame}
 \frametitle{}
\begin{itemize}
\item oxidizes substrate ( e. g pyruvate dehydrogenase) 
\end{itemize}
\end{frame}

%% page: 89
\begin{frame}
 \frametitle{Biochemistry 2}
carboxylase 
\end{frame}

%% page: 90
\begin{frame}
 \frametitle{}
\begin{itemize}
\item Add 1 carbon with the help of biotin ( e.g. pyruvae carboxylase) 
\end{itemize}
\end{frame}

%% page: 91
\begin{frame}
 \frametitle{Biochemistry 2}
Malate-aspartate shuttle 
\end{frame}

%% page: 92
\begin{frame}
 \frametitle{}
\begin{itemize}
\item heart and liver, aerobic metabolism, produce 32 ATP 
\end{itemize}
\end{frame}

%% page: 93
\begin{frame}
 \frametitle{Biochemistry 2}
glycerol-3-phosphate shuttle
\end{frame}

%% page: 94
\begin{frame}
 \frametitle{}
\begin{itemize}
\item muscle, aerobic metabolism, produce 30 ATP 
\end{itemize}
\end{frame}

%% page: 95
\begin{frame}
 \frametitle{Biochemistry 2}
ATP structure 
\end{frame}

%% page: 96
\begin{frame}
 \frametitle{}
\begin{itemize}
\item Base= Adenine, Triphosphate moiety, Ribose. 
\end{itemize}
\end{frame}

%% page: 97
\begin{frame}
 \frametitle{Biochemistry 2}
NADPH used in 
\end{frame}

%% page: 98
\begin{frame}
 \frametitle{}
\begin{itemize}
\item 1. Anabolic processes 2. Respiratory burst 3. P-450 4. Glutathione reductase  \\ anabolic processes ( steroid and fatty acid synthesis) 
\end{itemize}
\end{frame}

%% page: 99
\begin{frame}
 \frametitle{Biochemistry 2}
phosphorylation of glucose to yield glucose-6-phosphate 
\end{frame}

%% page: 100
\begin{frame}
 \frametitle{}
\begin{itemize}
\item 1st step of glycolysis, first step of glycogen syn in the liver. Catalyzed by hexokinase or glucokinase ( location depend) 
\end{itemize}
\end{frame}

%% page: 101
\begin{frame}
 \frametitle{Biochemistry 2}
glucokinase 
\end{frame}

%% page: 102
\begin{frame}
 \frametitle{}
\begin{itemize}
\item has high Vmax, it cannot be satisfied. 
\end{itemize}
\end{frame}

%% page: 103
\begin{frame}
 \frametitle{Biochemistry 2}
Glucose 
\end{frame}

%% page: 104
\begin{frame}
 \frametitle{}
\begin{itemize}
\item produce= glucose-6-phosphate, enzyme= hexokinase/ glucokinase, need ATP 
\end{itemize}
\end{frame}

%% page: 105
\begin{frame}
 \frametitle{Biochemistry 2}
fructose-6-P
\end{frame}

%% page: 106
\begin{frame}
 \frametitle{}
\begin{itemize}
\item produce= fructose-1,6-BP. Enzyme= phosphofructokinase-1, increase= AMP, fructose-2,6-BP. Inhibit= ATP and Citrate, need ATP 
\end{itemize}
\end{frame}

%% page: 107
\begin{frame}
 \frametitle{Biochemistry 2}
1,3-BPG 
\end{frame}

%% page: 108
\begin{frame}
 \frametitle{}
\begin{itemize}
\item procude= 3-PG, enzyme= phosphoglycerate kinase, produce ATP 
\end{itemize}
\end{frame}

%% page: 109
\begin{frame}
 \frametitle{Biochemistry 2}
phosphoenolpyruvate 
\end{frame}

%% page: 110
\begin{frame}
 \frametitle{}
\begin{itemize}
\item produce=pyruvate, enzyme= pyruvate kinase, increase= fructose-1,6-BP, decrease= ATP, and Alanine. 
\end{itemize}
\end{frame}

%% page: 111
\begin{frame}
 \frametitle{Biochemistry 2}
pyruvate 
\end{frame}

%% page: 112
\begin{frame}
 \frametitle{}
\begin{itemize}
\item produce= acetyl-coA, enzyme= pyruvate dehydrogenase, decrease= ATP, NADH, and acetyl-coA. 
\end{itemize}
\end{frame}

%% page: 113
\begin{frame}
 \frametitle{Biochemistry 2}
FBPase-2 and PFK-2 
\end{frame}

%% page: 114
\begin{frame}
 \frametitle{}
\begin{itemize}
\item are part of the same complex but respond in opposite manners to phosphorylation by protein kinase A 
\end{itemize}
\end{frame}

%% page: 115
\begin{frame}
 \frametitle{Biochemistry 2}
Fasting PFK-2
\end{frame}

%% page: 116
\begin{frame}
 \frametitle{}
\begin{itemize}
\item Fasting state= increase glucagon== increase cAMP== increase protein kinase A = increase FBPase-2, decrease PFK-2 \\ PFK-2 increase fructose-2,6-bisphosphate, stimulate PFK-1, increase glycolysis 
\end{itemize}
\end{frame}

%% page: 117
\begin{frame}
 \frametitle{Biochemistry 2}
fed state of PFK-2
\end{frame}

%% page: 118
\begin{frame}
 \frametitle{}
\begin{itemize}
\item increase insulin= decrease cAMP = decrease protein kinase A = decrease FBPase-2, increase PFK-2 
\end{itemize}
\end{frame}

%% page: 119
\begin{frame}
 \frametitle{Biochemistry 2}
pyruvate metabolism 
\end{frame}

%% page: 120
\begin{frame}
 \frametitle{}
\begin{itemize}
\item pyruvate ---Lactate ( H+, NADH and LDH needed, NAD+ produced)= .End of anaerobic glycolysis (major pathway in RBCs, leukocytes, kidney medulla, lens, testes, and cornea).   Pyruvate --- acetyl-coA, ( NAD+, PDH, B1 needed ) = transition from glycolysis to TCA cycle. 
\end{itemize}
\end{frame}

%% page: 121
\begin{frame}
 \frametitle{Biochemistry 2}
pyruvate to alanine 
\end{frame}

%% page: 122
\begin{frame}
 \frametitle{}
\begin{itemize}
\item ALT needed. Alanine carries amino
groups to the liver from
Muscle
\end{itemize}
\end{frame}

%% page: 123
\begin{frame}
 \frametitle{Biochemistry 2}
pyruvate to oxaloactetate 
\end{frame}

%% page: 124
\begin{frame}
 \frametitle{}
\begin{itemize}
\item PC,biotin, CO2, and ATP needed. Oxaloacetate can replenish
TCA cycle or be used in
Gluconeogenesis
\end{itemize}
\end{frame}

%% page: 125
\begin{frame}
 \frametitle{Biochemistry 2}
cori cycle 
\end{frame}

%% page: 126
\begin{frame}
 \frametitle{}
\begin{itemize}
\item muscle/RBC through anaerobic glycolysis to produce lactate, lactate goes to liver, in liver produce glucose, send to muscle/RBC 
\end{itemize}
\end{frame}

%% page: 127
\begin{frame}
 \frametitle{Biochemistry 2}
pyruvate to acetyl-coA
\end{frame}

%% page: 128
\begin{frame}
 \frametitle{}
\begin{itemize}
\item Produce 1 NADH, 1 CO2. PDH needed. 
\end{itemize}
\end{frame}

%% page: 129
\begin{frame}
 \frametitle{Biochemistry 2}
TCA cycle 
\end{frame}

%% page: 130
\begin{frame}
 \frametitle{}
\begin{itemize}
\item Produces 3 NADH, 1 FADH2, 2 CO2, 1 GTP per acetyl- CoA = 12 ATP/acetyl-CoA (2${\times}$ everything per glucose).
\end{itemize}
\end{frame}

%% page: 131
\begin{frame}
 \frametitle{Biochemistry 2}
${\alpha}$-ketoglutarate dehydrogenase
\end{frame}

%% page: 132
\begin{frame}
 \frametitle{}
\begin{itemize}
\item requires same cofactors as the pyruvate dehydrogenase complex (B1, B2, B3, B5, lipoic acid).
\end{itemize}
\end{frame}

%% page: 133
\begin{frame}
 \frametitle{Biochemistry 2}
NADH produced in TCA cycle 
\end{frame}

%% page: 134
\begin{frame}
 \frametitle{}
\begin{itemize}
\item 1. isocitrate to a-ketoglutarate ( CO2 produced also) 2. a-ketoglutarate to succinyl-coA ( CO2 produced also) 3. malate to oxaloacetate 
\end{itemize}
\end{frame}

%% page: 135
\begin{frame}
 \frametitle{Biochemistry 2}
FADH2 produced 
\end{frame}

%% page: 136
\begin{frame}
 \frametitle{}
\begin{itemize}
\item succinate to fumarate
\end{itemize}
\end{frame}

%% page: 137
\begin{frame}
 \frametitle{Biochemistry 2}
GTP produced 
\end{frame}

%% page: 138
\begin{frame}
 \frametitle{}
\begin{itemize}
\item Succinyl-coA to succinate  ( CoA produced) 
\end{itemize}
\end{frame}

%% page: 139
\begin{frame}
 \frametitle{Biochemistry 2}
Electron transport inhibitors
\end{frame}

%% page: 140
\begin{frame}
 \frametitle{}
\begin{itemize}
\item Rotenone, CN–, antimycin A, CO.
\end{itemize}
\end{frame}

%% page: 141
\begin{frame}
 \frametitle{Biochemistry 2}
ATPase inhibitors
\end{frame}

%% page: 142
\begin{frame}
 \frametitle{}
\begin{itemize}
\item Oligomycin.
\end{itemize}
\end{frame}

%% page: 143
\begin{frame}
 \frametitle{Biochemistry 2}
Uncoupling agents
\end{frame}

%% page: 144
\begin{frame}
 \frametitle{}
\begin{itemize}
\item aspirin, 2,4-DNP, aspirin, and thermogenin in brown fat.
\end{itemize}
\end{frame}

%% page: 145
\begin{frame}
 \frametitle{Biochemistry 2}
Electron transport
\end{frame}

%% page: 146
\begin{frame}
 \frametitle{}
\begin{itemize}
\item 1 NADH ${\rightarrow}$ 3 ATP; 1 FADH2 ${\rightarrow}$ 2 ATP. \\ NADH electrons from glycolysis and TCA cycle enter mitochondria via the malate-aspartate or glycerol-3-phosphate shuttle. FADH2 electrons are transferred to complexII ( at a lower energy level than NADH). 
\end{itemize}
\end{frame}

%% page: 147
\begin{frame}
 \frametitle{Biochemistry 2}
Pyruvate carboxylase
\end{frame}

%% page: 148
\begin{frame}
 \frametitle{}
\begin{itemize}
\item Requires biotin, ATP. Activated by acetyl-CoA.
\end{itemize}
\end{frame}

%% page: 149
\begin{frame}
 \frametitle{Biochemistry 2}
PEP carboxykinase
\end{frame}

%% page: 150
\begin{frame}
 \frametitle{}
\begin{itemize}
\item In cytosol. Oxaloacetate ${\rightarrow}$ phosphoenolpyruvate. \\ Requires GTP.
\end{itemize}
\end{frame}

%% page: 151
\begin{frame}
 \frametitle{Biochemistry 2}
Pathway Produces Fresh Glucose.
\end{frame}

%% page: 152
\begin{frame}
 \frametitle{}
\begin{itemize}
\item Pyruvate carboxylase, PEP carboxykinase, Fructose-1,6- bisphosphatase, Glucose-6- phosphatase
\end{itemize}
\end{frame}

%% page: 153
\begin{frame}
 \frametitle{Biochemistry 2}
muscle cannot gluconeogenesis
\end{frame}

%% page: 154
\begin{frame}
 \frametitle{}
\begin{itemize}
\item  it lacks glucose-6-phosphatase
\end{itemize}
\end{frame}

%% page: 155
\begin{frame}
 \frametitle{Biochemistry 2}
HMP shunt, pentose phosphate pathway
\end{frame}

%% page: 156
\begin{frame}
 \frametitle{}
\begin{itemize}
\item Produces NADPH from glucose-6-phosphate, yields ribose for nucleotide syn, and glycolytic intermediates, 2 distinct phases, both occur in cytoplasm, No ATP is used or produced. 
\end{itemize}
\end{frame}

%% page: 157
\begin{frame}
 \frametitle{Biochemistry 2}
Oxidative phase of HMP 
\end{frame}

%% page: 158
\begin{frame}
 \frametitle{}
\begin{itemize}
\item Glucose-6-phosphate dehydrogenase \\ Gucose-6-pi---- 2 NADPH, ribulose-5-Pi, CO2,  \\ NADPH (for fatty acid and steroid synthesis, glutathione reduction, and cytochrome P-450)
\end{itemize}
\end{frame}

%% page: 159
\begin{frame}
 \frametitle{Biochemistry 2}
nonoxidative phase of HMP 
\end{frame}

%% page: 160
\begin{frame}
 \frametitle{}
\begin{itemize}
\item Transketolases (require thiamine) \\ ribulose-5-Pi----- ribulose-6-Pi, G3P, F6P. 
\end{itemize}
\end{frame}

%% page: 161
\begin{frame}
 \frametitle{Biochemistry 2}
respiratory burst ( oxidative burst)
\end{frame}

%% page: 162
\begin{frame}
 \frametitle{}
\begin{itemize}
\item involves the activation of membrane-bound NADPH oxidase ( in neutrophils, macrophages). Results in the rapid release of reactive oxygen intermediates ( ROIs) 
\end{itemize}
\end{frame}

%% page: 163
\begin{frame}
 \frametitle{Biochemistry 2}
O2 to O2 *
\end{frame}

%% page: 164
\begin{frame}
 \frametitle{}
\begin{itemize}
\item NADPH oxidase. Deficiency = chronic granulomatous disease. 
\end{itemize}
\end{frame}

%% page: 165
\begin{frame}
 \frametitle{Biochemistry 2}
O2 * to H2O2 
\end{frame}

%% page: 166
\begin{frame}
 \frametitle{}
\begin{itemize}
\item superoxide dismutase
\end{itemize}
\end{frame}

%% page: 167
\begin{frame}
 \frametitle{Biochemistry 2}
H2O2 to HOCl
\end{frame}

%% page: 168
\begin{frame}
 \frametitle{}
\begin{itemize}
\item myeloperoxidase
\end{itemize}
\end{frame}

%% page: 169
\begin{frame}
 \frametitle{Biochemistry 2}
H2O2 plus GSH
\end{frame}

%% page: 170
\begin{frame}
 \frametitle{}
\begin{itemize}
\item catalase/Glutathione peroxidase, 
\end{itemize}
\end{frame}

%% page: 171
\begin{frame}
 \frametitle{Biochemistry 2}
GSSG + NADPH 
\end{frame}

%% page: 172
\begin{frame}
 \frametitle{}
\begin{itemize}
\item produce GSH + NADP+, glutathione reductase. 
\end{itemize}
\end{frame}

%% page: 173
\begin{frame}
 \frametitle{Biochemistry 2}
NADP+ + G6P
\end{frame}

%% page: 174
\begin{frame}
 \frametitle{}
\begin{itemize}
\item produce NADPH and 6PG, enzyme = G6PD, 
\end{itemize}
\end{frame}

%% page: 175
\begin{frame}
 \frametitle{Biochemistry 2}
WBCs of  patients with Chronic granulomatous disease
\end{frame}

%% page: 176
\begin{frame}
 \frametitle{}
\begin{itemize}
\item can utilize H2O2 generated by invading organisms and convert it to ROIS. Patients are at increase risk for infection by catalase-positive species ( e. G. S. aureus, aspergillus) because they neutralize their own H2O2, leaving  WBC without ROIs for fighting infections. 
\end{itemize}
\end{frame}

%% page: 177
\begin{frame}
 \frametitle{Biochemistry 2}
Infection + G6P deficiency + hemolysis in RBC
\end{frame}

%% page: 178
\begin{frame}
 \frametitle{}
\begin{itemize}
\item free radical generated via inflammatory response can diffuse into RBCs and cause oxidative damage.  \\ G6PD deficiency is more prevalent among blacks. \\ X-linked recessive disorder. ${\uparrow}$ malarial resistance.
\end{itemize}
\end{frame}

%% page: 179
\begin{frame}
 \frametitle{Biochemistry 2}
Galactokinase deficiency symptoms 
\end{frame}

%% page: 180
\begin{frame}
 \frametitle{}
\begin{itemize}
\item galaxtose appears in blood and urine, infantile cataracts. May initially present as failure to track objects or to develop a social smile
\end{itemize}
\end{frame}

%% page: 181
\begin{frame}
 \frametitle{Biochemistry 2}
Lactase deficiency
\end{frame}

%% page: 182
\begin{frame}
 \frametitle{}
\begin{itemize}
\item Age-dependent and/or hereditary lactose intolerance (blacks, Asians) due to loss of brush-border enzyme. \\ Symptoms: bloating, cramps, osmotic diarrhea. Treatment: avoid milk or add lactase pills to diet.
\end{itemize}
\end{frame}

%% page: 183
\begin{frame}
 \frametitle{Biochemistry 2}
sorbitol 
\end{frame}

%% page: 184
\begin{frame}
 \frametitle{}
\begin{itemize}
\item from glucose through enzyme ( aldose reductase), then enzyme ( sorbitol dehydrogenase) to fructose in organ ( liver, ovaries, and seminal vesicles).  \\ in schwann cells, lens, retina, and kidney, no sorbitol dehydrogenase, sorbitol accumulation, osmotic pressure, then water into cell. Leads to cataracts, retinopathy, and peripheral neuropathy in prolonged hyperglycemic diabets. So does fructose, and galactose like glucose. 
\end{itemize}
\end{frame}

%% page: 185
\begin{frame}
 \frametitle{Biochemistry 2}
Glucogenic/ketogenic
\end{frame}

%% page: 186
\begin{frame}
 \frametitle{}
\begin{itemize}
\item Ile, Phe, Tyr, Thr.
\end{itemize}
\end{frame}

%% page: 187
\begin{frame}
 \frametitle{Biochemistry 2}
Glucogenic
\end{frame}

%% page: 188
\begin{frame}
 \frametitle{}
\begin{itemize}
\item Met, Val, Arg, His. \\ Only L-form amino acids are found in proteins.
\end{itemize}
\end{frame}

%% page: 189
\begin{frame}
 \frametitle{Biochemistry 2}
Basic
\end{frame}

%% page: 190
\begin{frame}
 \frametitle{}
\begin{itemize}
\item Arg and His are required during periods of growth. \\ Arg and Lys are ${\uparrow}$ in histones, which bind negatively charged DNA. \\ Arg is most basic.His has no charge at body pH.
\end{itemize}
\end{frame}

%% page: 191
\begin{frame}
 \frametitle{Biochemistry 2}
Urea cycle 
\end{frame}

%% page: 192
\begin{frame}
 \frametitle{}
\begin{itemize}
\item amino acid catabolism results in the formation of common metabolites ( e. g = pyruvate, acetyl-CoA). Urea ( NH2-CO-NH2), NH2 from ( NH4+ and aspartate), CO from CO2
\end{itemize}
\end{frame}

%% page: 193
\begin{frame}
 \frametitle{Biochemistry 2}
Ordinarily, careless crappers Are Also Frivolous About Urination 
\end{frame}

%% page: 194
\begin{frame}
 \frametitle{}
\begin{itemize}
\item Ornithine, Carbamoyl phosphate, Citrulline, Aspartate, Argininosuccinate, fumarate, Arginine, Urea. 
\end{itemize}
\end{frame}

%% page: 195
\begin{frame}
 \frametitle{Biochemistry 2}
Transport of ammonium by alanine and glutamine
\end{frame}

%% page: 196
\begin{frame}
 \frametitle{}
\begin{itemize}
\item  Glutamate + pyruvate== a-ketoglutarate + alanine. Amino acid + a-ketoglutarate== a-ketoacids + glutarate.  \\ in liver, produce glucose, and Urea from glutamate 
\end{itemize}
\end{frame}

%% page: 197
\begin{frame}
 \frametitle{Biochemistry 2}
Ammonia intoxication
\end{frame}

%% page: 198
\begin{frame}
 \frametitle{}
\begin{itemize}
\item tremor, slurring of speech, somnolence, vomiting, cerebral edema, blurring of vision.
\end{itemize}
\end{frame}

%% page: 199
\begin{frame}
 \frametitle{Biochemistry 2}
ornithine transcarbamoylase deficiency 
\end{frame}

%% page: 200
\begin{frame}
 \frametitle{}
\begin{itemize}
\item Most common urea cycle disorder. X-linked recessive ( other urea cylce enzyme deficiency are autosomal recessive). Interferes with the body's ability to eliminate ammonia. Often evident in the first few days of life. But may present with late onset. Excess carbomoyl phosphate is converted to orotic acid ( part of pyrimidine syn pathway). Findings= orotic acid in blood and urine, decrease BUN, symptoms of hyperammonemia. 
\end{itemize}
\end{frame}

%% page: 201
\begin{frame}
 \frametitle{Biochemistry 2}
breakdown products via MAO and COMT 
\end{frame}

%% page: 202
\begin{frame}
 \frametitle{}
\begin{itemize}
\item  Dopamine= HVA, Norepinephrine= VMA,  Epinephrine= metanephrine. 
\end{itemize}
\end{frame}

%% page: 203
\begin{frame}
 \frametitle{Biochemistry 2}
maternal PKU 
\end{frame}

%% page: 204
\begin{frame}
 \frametitle{}
\begin{itemize}
\item lack of proper dietary therapy during pregnancy. Findings in infant= microcephaly, mental retardation, growth retardation, congenital heart defects. 
\end{itemize}
\end{frame}

%% page: 205
\begin{frame}
 \frametitle{Biochemistry 2}
Phenylketones–
\end{frame}

%% page: 206
\begin{frame}
 \frametitle{}
\begin{itemize}
\item phenylacetate, \\ phenyllactate, and \\ phenylpyruvate.
\end{itemize}
\end{frame}

%% page: 207
\begin{frame}
 \frametitle{Biochemistry 2}
PKU 
\end{frame}

%% page: 208
\begin{frame}
 \frametitle{}
\begin{itemize}
\item Autosomal-recessive disease. \\ Incidence ${\simeq}$ 1:10,000. Disorder of aromatic amino \\ acid metabolism ${\rightarrow}$ musty body odor.
\end{itemize}
\end{frame}

%% page: 209
\begin{frame}
 \frametitle{Biochemistry 2}
ocular albinism 
\end{frame}

%% page: 210
\begin{frame}
 \frametitle{}
\begin{itemize}
\item X-linked recessive 
\end{itemize}
\end{frame}

%% page: 211
\begin{frame}
 \frametitle{Biochemistry 2}
albinism 
\end{frame}

%% page: 212
\begin{frame}
 \frametitle{}
\begin{itemize}
\item Lack of melanin results in an ${\uparrow}$ risk of skin cancer. \\ Variable inheritance due to locus heterogeneity.
\end{itemize}
\end{frame}

%% page: 213
\begin{frame}
 \frametitle{Biochemistry 2}
Homocystinuria
\end{frame}

%% page: 214
\begin{frame}
 \frametitle{}
\begin{itemize}
\item result in excess homocysteine, and cysteine becomes essential. \\ Can cause mental retardation, osteoporosis, tall stature, kyphosis, lens subluxation (downward and inward), and atherosclerosis (stroke and MI).
\end{itemize}
\end{frame}

%% page: 215
\begin{frame}
 \frametitle{Biochemistry 2}
cysteine metabolism 
\end{frame}

%% page: 216
\begin{frame}
 \frametitle{}
\begin{itemize}
\item methionine get rid of CH3 vis Sam = homocysteine. Homocysteine to cystathionine through enzyme ( cystathionine synthase with B6), then to cysteine.  \\ Homocysteine with CH3 THF through enzyme ( homocysteine methyltransferase with B12) to THF and Methionine. 
\end{itemize}
\end{frame}

%% page: 217
\begin{frame}
 \frametitle{Biochemistry 2}
Cystinuria
\end{frame}

%% page: 218
\begin{frame}
 \frametitle{}
\begin{itemize}
\item Treat with acetazolamide to alkalinize the urine. \\ Cystine is made of 2 cysteines connected by a disulfide bond. \\ Autosomal-recessive disease.
\end{itemize}
\end{frame}

%% page: 219
\begin{frame}
 \frametitle{Biochemistry 2}
I Love Vermont maple syrup.
\end{frame}

%% page: 220
\begin{frame}
 \frametitle{}
\begin{itemize}
\item blocked degradation of branched amino acids (Ile, Val, Leu) due to ${\downarrow}$ ${\alpha}$-ketoacid dehydrogenase.
\end{itemize}
\end{frame}

%% page: 221
\begin{frame}
 \frametitle{Biochemistry 2}
hartnup disease 
\end{frame}

%% page: 222
\begin{frame}
 \frametitle{}
\begin{itemize}
\item an autosomal recessive disorder. Defective neutral amino acid transporter on renal and intestinal epithelial cells. Causes tryptophan excretion in urine and absorption from the gut. Leads to pellagra. 
\end{itemize}
\end{frame}

%% page: 223
\begin{frame}
 \frametitle{Biochemistry 2}
Insulin with glycogen 
\end{frame}

%% page: 224
\begin{frame}
 \frametitle{}
\begin{itemize}
\item Insulin dephosphorylates (${\downarrow}$ cAMP ${\rightarrow}$ ${\downarrow}$ PKA= protein kinase A). \\ insulin bind with receptor tyrosine kinase dimerizes, through protein phosphatase, dephosphate of glycogen phosphorylase and glycogen phosphrylase kinase, make them inactive, stop break down glycogen. 
\end{itemize}
\end{frame}

%% page: 225
\begin{frame}
 \frametitle{Biochemistry 2}
glucagon and epinephrine with glycogen 
\end{frame}

%% page: 226
\begin{frame}
 \frametitle{}
\begin{itemize}
\item glucagon from liver, Epi from liver and muscle through Adenylyl cyclase, increase cAMP, increase protein kinase A, phosphalase Glycogen phosphorylase kinase and glycoen phosphorylase, make them active, break down glycogen.  \\ Ca++ /Calmodulin in muscle activates phosphorylase kinase==glycogenolysis. 
\end{itemize}
\end{frame}

%% page: 227
\begin{frame}
 \frametitle{Biochemistry 2}
glycogen 
\end{frame}

%% page: 228
\begin{frame}
 \frametitle{}
\begin{itemize}
\item Branches have ${\alpha}$ (1,6) bonds; linkages have ${\alpha}$ (1,4) bonds. \\  UDP-glucose pyrophosphorylase  Glycogen synthase  Branching enzyme  Glycogen phosphorylase \\  Debranching enzyme \\ A small amount of glycogen is degraded in lysosomes by ${\alpha}$-1,4- glucosidase \\ glucose-1-phosphate through UDP-glucose pyrophosphorylase to UDP-glucose, then through enzyme ( glycogen synthase) to storage form of glycogen 
\end{itemize}
\end{frame}

%% page: 229
\begin{frame}
 \frametitle{Biochemistry 2}
Glucose-6-phosphate 
\end{frame}

%% page: 230
\begin{frame}
 \frametitle{}
\begin{itemize}
\item through enzyme ( glucose-6-phosphatase) to glucose +Pi \\ Von Gierke’s disease = enzyme deficiency 
\end{itemize}
\end{frame}

%% page: 231
\begin{frame}
 \frametitle{Biochemistry 2}
Pompe’s trashes the Pump (heart, liver, and muscle).
\end{frame}

%% page: 232
\begin{frame}
 \frametitle{}
\begin{itemize}
\item Lysosomal ${\alpha}$-1,4- glucosidase (acid maltase) deficiency 
\end{itemize}
\end{frame}

%% page: 233
\begin{frame}
 \frametitle{Biochemistry 2}
Debranching enzyme ${\alpha}$-1,6-glucosidase deficiency 
\end{frame}

%% page: 234
\begin{frame}
 \frametitle{}
\begin{itemize}
\item Gluconeogenesis is intact.normal blood lactate levels
\end{itemize}
\end{frame}

%% page: 235
\begin{frame}
 \frametitle{Biochemistry 2}
Lysosomal storage diseases XR 
\end{frame}

%% page: 236
\begin{frame}
 \frametitle{}
\begin{itemize}
\item Fabry’s disease of defiency =${\alpha}$-galactosidase A = Sphingolipidoses= accumulate Ceramide	trihexoside \\ Hunter’s syndrome = Iduronate sulfatase deficiency =  Mucopolysaccharidoses = accumulate Heparan sulfate,	 dermatan sulfate
\end{itemize}
\end{frame}

%% page: 237
\begin{frame}
 \frametitle{Biochemistry 2}
fatty acid syn 
\end{frame}

%% page: 238
\begin{frame}
 \frametitle{}
\begin{itemize}
\item start from acetyl-coA in mitochondrial, through citrate shuttle to cytoplasm, then + CO2 ( biotin) to malonyl-CoA, then make 16 C FA ( palmitate).  \\ SYtrate = SYnthesis.
\end{itemize}
\end{frame}

%% page: 239
\begin{frame}
 \frametitle{Biochemistry 2}
fatty acid degradation 
\end{frame}

%% page: 240
\begin{frame}
 \frametitle{}
\begin{itemize}
\item Fatty acid + CoA through enyzme ( fatty acid CoA synthetase) to Acyl-CoA ( in cytoplasm), then through carnitine shuttle to mitochondrial, through b-oxidation breakdown to acetyl-CoA groups, finally get ketone bodies + productor to TCA cycle.  \\ Fatty acid degradation occurs where its products will be consumed—in the mitochondrion.
\end{itemize}
\end{frame}

%% page: 241
\begin{frame}
 \frametitle{Biochemistry 2}
Acyl-CoA dehydrogenase deficiency:
\end{frame}

%% page: 242
\begin{frame}
 \frametitle{}
\begin{itemize}
\item  ${\uparrow}$ dicarboxylic acids, ${\downarrow}$ glucose and ketones.
\end{itemize}
\end{frame}

%% page: 243
\begin{frame}
 \frametitle{Biochemistry 2}
Carnitine deficiency:
\end{frame}

%% page: 244
\begin{frame}
 \frametitle{}
\begin{itemize}
\item  inability to utilize LCFAs and toxic accumulation. \\ causes weakness, hypotonia, hypoketotic, and hypoglycemia 
\end{itemize}
\end{frame}

%% page: 245
\begin{frame}
 \frametitle{Biochemistry 2}
Ketone bodies
\end{frame}

%% page: 246
\begin{frame}
 \frametitle{}
\begin{itemize}
\item Made from HMG-CoA. Ketone bodies are metabolized \\ by the brain to 2 molecules of acetyl-CoA. \\ In liver, fatty acid and amino acids are metabolized to acetoacetate and ${\beta}$-hydroxybutyrate (to be used in muscle and brain). \\ In prolonged starvation and diabetic ketoacidosis, oxaloacetate is depleted for gluconeogenesis. In alcoholism, excess NADH shunts oxaloacetate to malate. Both processes stall the TCA cycle, which shunts glucose and FFA to ketone bodies. 
\end{itemize}
\end{frame}

%% page: 247
\begin{frame}
 \frametitle{Biochemistry 2}
Cholesterol synthesis
\end{frame}

%% page: 248
\begin{frame}
 \frametitle{}
\begin{itemize}
\item Rate-limiting step is catalyzed by HMG-CoA reductase, which converts HMG-CoA to mevalonate. 2/3 of plasma cholesterol is esterified by lecithin-cholesterol acyltransferase (LCAT).
\end{itemize}
\end{frame}

%% page: 249
\begin{frame}
 \frametitle{Biochemistry 2}
Essential fatty acids
\end{frame}

%% page: 250
\begin{frame}
 \frametitle{}
\begin{itemize}
\item Linoleic and linolenic acids. \\ Arachidonic acid, if linoleic acid is absent. \\ Eicosanoids are dependent on essential fatty acids.
\end{itemize}
\end{frame}

%% page: 251
\begin{frame}
 \frametitle{Biochemistry 2}
100-meter sprint (seconds) energy use 
\end{frame}

%% page: 252
\begin{frame}
 \frametitle{}
\begin{itemize}
\item stored ATP, creatine phosphate, anaerobic  glycolysis. \\ 1 g protein or carbohydrate = 4 kcal. 1 g fat= 9 kcal 
\end{itemize}
\end{frame}

%% page: 253
\begin{frame}
 \frametitle{Biochemistry 2}
1000-meter run (minutes)
\end{frame}

%% page: 254
\begin{frame}
 \frametitle{}
\begin{itemize}
\item + oxidative phosphorylation.
\end{itemize}
\end{frame}

%% page: 255
\begin{frame}
 \frametitle{Biochemistry 2}
Marathon (hours)
\end{frame}

%% page: 256
\begin{frame}
 \frametitle{}
\begin{itemize}
\item Glycogen and FFA oxidation; glucose conserved for final sprinting.
\end{itemize}
\end{frame}

%% page: 257
\begin{frame}
 \frametitle{Biochemistry 2}
Fasting and starvation
\end{frame}

%% page: 258
\begin{frame}
 \frametitle{}
\begin{itemize}
\item Priorities are to supply sufficient glucose to brain and RBCs and to preserve protein
\end{itemize}
\end{frame}

%% page: 259
\begin{frame}
 \frametitle{Biochemistry 2}
Days 1–3
\end{frame}

%% page: 260
\begin{frame}
 \frametitle{}
\begin{itemize}
\item lood glucose level maintained by: 1. Hepatic glycogenolysis and glucose release 2. Adipose release of FFA 3. Muscle and liver shifting fuel use from glucose \\ to FFA 4. Hepatic gluconeogenesis from peripheral tissue \\ lactate and alanine, and from adipose tissue glycerol and propionyl-CoA from odd-chain FFA metabolism (the only triacylglycerol components that can contribute to gluconeogenesis)
\end{itemize}
\end{frame}

%% page: 261
\begin{frame}
 \frametitle{Biochemistry 2}
After day 3
\end{frame}

%% page: 262
\begin{frame}
 \frametitle{}
\begin{itemize}
\item Muscle protein loss is maintained by hepatic formation of ketone bodies, supplying the brain and heart.
\end{itemize}
\end{frame}

%% page: 263
\begin{frame}
 \frametitle{Biochemistry 2}
After several weeks
\end{frame}

%% page: 264
\begin{frame}
 \frametitle{}
\begin{itemize}
\item Ketone bodies become main source of energy for brain, so less muscle protein is degraded than during days 1–3. Survival time is determined by amount of fat stores. After this is depleted, vital protein degradation accelerates, leading to organ failure and death.
\end{itemize}
\end{frame}

%% page: 265
\begin{frame}
 \frametitle{Biochemistry 2}
Pancreatic lipase
\end{frame}

%% page: 266
\begin{frame}
 \frametitle{}
\begin{itemize}
\item degradation of dietary TG in small intestine.
\end{itemize}
\end{frame}

%% page: 267
\begin{frame}
 \frametitle{Biochemistry 2}
Lipoprotein lipase (LPL)
\end{frame}

%% page: 268
\begin{frame}
 \frametitle{}
\begin{itemize}
\item degradation of TG circulating in chylomicrons and VLDLs.
\end{itemize}
\end{frame}

%% page: 269
\begin{frame}
 \frametitle{Biochemistry 2}
Hepatic TG lipase (HL)
\end{frame}

%% page: 270
\begin{frame}
 \frametitle{}
\begin{itemize}
\item degradation of TG remaining in IDL.
\end{itemize}
\end{frame}

%% page: 271
\begin{frame}
 \frametitle{Biochemistry 2}
Hormone-sensitive lipase
\end{frame}

%% page: 272
\begin{frame}
 \frametitle{}
\begin{itemize}
\item degradation of TG stored in adipocytes.
\end{itemize}
\end{frame}

%% page: 273
\begin{frame}
 \frametitle{Biochemistry 2}
Lecithin-cholesterol acyltransferase (LCAT)
\end{frame}

%% page: 274
\begin{frame}
 \frametitle{}
\begin{itemize}
\item catalyzes esterification of cholesterol
\end{itemize}
\end{frame}

%% page: 275
\begin{frame}
 \frametitle{Biochemistry 2}
Cholesterol ester transfer protein (CETP)
\end{frame}

%% page: 276
\begin{frame}
 \frametitle{}
\begin{itemize}
\item mediates transfer of cholesterol esters to
other lipoprotein particles.
\end{itemize}
\end{frame}

%% page: 277
\begin{frame}
 \frametitle{Biochemistry 2}
A-I–
\end{frame}

%% page: 278
\begin{frame}
 \frametitle{}
\begin{itemize}
\item Activates LCAT.
\end{itemize}
\end{frame}

%% page: 279
\begin{frame}
 \frametitle{Biochemistry 2}
B-100–
\end{frame}

%% page: 280
\begin{frame}
 \frametitle{}
\begin{itemize}
\item Binds to LDL receptor, mediates VLDL secretion.
\end{itemize}
\end{frame}

%% page: 281
\begin{frame}
 \frametitle{Biochemistry 2}
C-II––
\end{frame}

%% page: 282
\begin{frame}
 \frametitle{}
\begin{itemize}
\item Cofactor for lipoprotein lipase.
\end{itemize}
\end{frame}

%% page: 283
\begin{frame}
 \frametitle{Biochemistry 2}
B-48
\end{frame}

%% page: 284
\begin{frame}
 \frametitle{}
\begin{itemize}
\item Mediates chylomicron secretion.
\end{itemize}
\end{frame}

%% page: 285
\begin{frame}
 \frametitle{Biochemistry 2}
E–
\end{frame}

%% page: 286
\begin{frame}
 \frametitle{}
\begin{itemize}
\item Mediates Extra (remnant) uptake.
\end{itemize}
\end{frame}

%% page: 287
\begin{frame}
 \frametitle{Biochemistry 2}
Lipoprotein functions
\end{frame}

%% page: 288
\begin{frame}
 \frametitle{}
\begin{itemize}
\item Lipoproteins are composed of varying proportions of cholesterol, triglycerides, and phospholipids. \\ LDL transports cholesterol from liver to tissue; HDL transports it from periphery to liver.
\end{itemize}
\end{frame}

%% page: 289
\begin{frame}
 \frametitle{Biochemistry 2}
Chylomicron
\end{frame}

%% page: 290
\begin{frame}
 \frametitle{}
\begin{itemize}
\item Delivers dietary triglycerides to peripheral tissues. Delivers cholesterol to liver in the form of chylomicron remnants, which are mostly depleted of their triacylglycerols. Secreted by intestinal epithelial cells. Excess causes pancreatitis, lipemia retinalis, and eruptive xanthomas. \\ Apolipoproteins \\ B-48, A-IV, C-II, and E
\end{itemize}
\end{frame}

%% page: 291
\begin{frame}
 \frametitle{Biochemistry 2}
VLDL
\end{frame}

%% page: 292
\begin{frame}
 \frametitle{}
\begin{itemize}
\item Delivers hepatic triglycerides to peripheral tissues. Secreted by liver. Excess causes pancreatitis. \\ B-100, C-II, and E
\end{itemize}
\end{frame}

%% page: 293
\begin{frame}
 \frametitle{Biochemistry 2}
IDL
\end{frame}

%% page: 294
\begin{frame}
 \frametitle{}
\begin{itemize}
\item Formed in the degradation of VLDL. Delivers triglycerides and cholesterol to liver, where they are degraded to LDL. \\ B-100 and E
\end{itemize}
\end{frame}

%% page: 295
\begin{frame}
 \frametitle{Biochemistry 2}
LDL
\end{frame}

%% page: 296
\begin{frame}
 \frametitle{}
\begin{itemize}
\item Delivers hepatic cholesterol to peripheral tissues. Formed by lipoprotein lipase modification of VLDL in the peripheral tissue. Taken up by target cells via receptor-mediated endocytosis. Excess causes atherosclerosis, xanthomas, and arcus corneae. \\ B-100
\end{itemize}
\end{frame}

%% page: 297
\begin{frame}
 \frametitle{Biochemistry 2}
HDL
\end{frame}

%% page: 298
\begin{frame}
 \frametitle{}
\begin{itemize}
\item Mediates centripetal transport of cholesterol (reverse cholesterol transport, from periphery to liver). Acts as a repository for apoC and apoE (which are needed for chylomicron and VLDL metabolism). Secreted from both liver and intestine.
\end{itemize}
\end{frame}

%% page: 299
\begin{frame}
 \frametitle{Biochemistry 2}
hyperchylomicronemia
\end{frame}

%% page: 300
\begin{frame}
 \frametitle{}
\begin{itemize}
\item Chylomicrons increased, TG, cholesterol increased in blood level, Lipoprotein lipase deficiency or altered apolipoprotein C-II. Causes pancreatitis, hepatosplenomegaly, and eruptive/pruritic xanthomas but no increase of athrosclerosis 
\end{itemize}
\end{frame}

%% page: 301
\begin{frame}
 \frametitle{Biochemistry 2}
hypercholesterolemia
\end{frame}

%% page: 302
\begin{frame}
 \frametitle{}
\begin{itemize}
\item LDL increased, Cholesterol increased in blood level. Autosomal dominant, absent or decrease LDL receptors. Causes accelerated atherosclerosis, tendon ( achilles) xanthomas, and corneal arcus.  
\end{itemize}
\end{frame}

%% page: 303
\begin{frame}
 \frametitle{Biochemistry 2}
hypertriglyceridemia
\end{frame}

%% page: 304
\begin{frame}
 \frametitle{}
\begin{itemize}
\item VLDL increase, TG increased in blood level, hepatic overproduction of VLDL. Causes pancreatitis. 
\end{itemize}
\end{frame}

%% page: 305
\begin{frame}
 \frametitle{Biochemistry 2}
Abeta-lipoproteinemia 
\end{frame}

%% page: 306
\begin{frame}
 \frametitle{}
\begin{itemize}
\item hereditary inability to syn lipoproteins due to deficiency in apoB-100 and apoB-48. Autosomal recessive. Symptoms appear in the first few months of life. Intestinal biospy shows accumulation within enterocytes due to inability to export absorbed lipid as chylomicrons. Findings: failure to thrive, steatorrhea, acanthocytosis, ataxia, and night blindness. 
\end{itemize}
\end{frame}

%% page: 307
\begin{frame}
 \frametitle{Biochemistry 2}
Polymerase chain reaction (PCR)
\end{frame}

%% page: 308
\begin{frame}
 \frametitle{}
\begin{itemize}
\item Molecular biology laboratory procedure that is used to synthesize many copies of a desired fragment of DNA.
Steps: 1. DNA is denatured by heating to generate 2 separate strands 2. During cooling, excess premade DNA primers anneal to a specific sequence on each
strand to be amplified 3. Heat-stable DNA polymerase replicates the DNA sequence following each primer
These steps are repeated multiple times for DNA sequence amplification.
\end{itemize}
\end{frame}

%% page: 309
\begin{frame}
 \frametitle{Biochemistry 2}
agarose gel eletrophoresis 
\end{frame}

%% page: 310
\begin{frame}
 \frametitle{}
\begin{itemize}
\item used for size separation of PCR products ( smaller molecules travel further), compared against DNA ladder. 
\end{itemize}
\end{frame}

%% page: 311
\begin{frame}
 \frametitle{Biochemistry 2}
Southern blot
\end{frame}

%% page: 312
\begin{frame}
 \frametitle{}
\begin{itemize}
\item A DNA sample is electrophoresed on a gel and then transferred to a filter. The filter is then soaked in a denaturant and subsequently exposed to a labeled DNA probe that recognizes and anneals to its complementary strand. The resulting double- stranded labeled piece of DNA is visualized \\ when the filter is exposed to film.
\end{itemize}
\end{frame}

%% page: 313
\begin{frame}
 \frametitle{Biochemistry 2}
Northern blot
\end{frame}

%% page: 314
\begin{frame}
 \frametitle{}
\begin{itemize}
\item Similar technique, except that Northern blotting \\ involves radioactive DNA probe binding to \\ sample RNA.
\end{itemize}
\end{frame}

%% page: 315
\begin{frame}
 \frametitle{Biochemistry 2}
Western blot
\end{frame}

%% page: 316
\begin{frame}
 \frametitle{}
\begin{itemize}
\item Sample protein is separated via gel electrophoresis \\ and transferred to a filter. Labeled antibody is \\ used to bind to relevant protein.
\end{itemize}
\end{frame}

%% page: 317
\begin{frame}
 \frametitle{Biochemistry 2}
Microarrays
\end{frame}

%% page: 318
\begin{frame}
 \frametitle{}
\begin{itemize}
\item Thousands of nucleic acid sequences are arranged \\ in grids on glass or silicon. DNA or RNA probes are hybridized to the chip, and a scanner detects the relative amounts of complementary binding. \\ used to profile gene expression levels or to detect single nucleotide polymorphisms (SNPs)
\end{itemize}
\end{frame}

%% page: 319
\begin{frame}
 \frametitle{Biochemistry 2}
Enzyme-linked immunosorbent assay (ELISA)
\end{frame}

%% page: 320
\begin{frame}
 \frametitle{}
\begin{itemize}
\item A rapid immunologic technique testing for antigen-antibody reactivity.
Patient’s blood sample is probed with either 1. Test antigen (coupled to color-generating
enzyme)––to see if immune system
recognizes it; or 2. Test antibody (coupled to color-generating
enzyme)––to see if a certain antigen is
present If the target substance is present in the sample,
the test solution will have an intense color reaction, indicating a positive test result.
\end{itemize}
\end{frame}

%% page: 321
\begin{frame}
 \frametitle{Biochemistry 2}
Fluorescence in situ hybridization (FISH)
\end{frame}

%% page: 322
\begin{frame}
 \frametitle{}
\begin{itemize}
\item Fluorescent probe binds to specific gene site of interest. Specific localization of genes and direct visualization of anomalies (e.g., microdeletions)
at molecular level ( when deletion is too small to be visualized by karyotype.) \\ Fluorescence = gene is present. No fluorescence= gene has been deleted. 
\end{itemize}
\end{frame}

%% page: 323
\begin{frame}
 \frametitle{Biochemistry 2}
palindromic sequences
\end{frame}

%% page: 324
\begin{frame}
 \frametitle{}
\begin{itemize}
\item the sequence on 1 strand reads the same in the same direction on the complementary strand),
\end{itemize}
\end{frame}

%% page: 325
\begin{frame}
 \frametitle{Biochemistry 2}
Cloning methods
\end{frame}

%% page: 326
\begin{frame}
 \frametitle{}
\begin{itemize}
\item Cloning is the production of a recombinant DNA molecule that is self-perpetuating. 1. DNA fragments are inserted into bacterial plasmids that contain antibiotic \\ resistance genes. These plasmids can be selected for by using media containing the antibiotic, and amplified. Restriction enzymes ligate DNA at 4- to 6-bp palindromic sequences, allowing for insertion of a fragment into a plasmid. \\ 2. Tissue mRNA is isolated and exposed to reverse transcriptase, forming a \\ cDNA (lacks introns) library.
\end{itemize}
\end{frame}

%% page: 327
\begin{frame}
 \frametitle{Biochemistry 2}
Sanger DNA sequencing
\end{frame}

%% page: 328
\begin{frame}
 \frametitle{}
\begin{itemize}
\item Dideoxynucleotides halt DNA polymerization at each base, generating sequences of various lengths that encompass the entire original sequence. The terminated fragments are electrophoresed and the original sequence can be deduced.
\end{itemize}
\end{frame}

%% page: 329
\begin{frame}
 \frametitle{Biochemistry 2}
Transgenic strategies in mice involve:
\end{frame}

%% page: 330
\begin{frame}
 \frametitle{}
\begin{itemize}
\item 1. Random insertion of gene into mouse genome (constitutive). 2. Targeted insertion or deletion of gene through homologous recombination with
mouse gene (constitutive). Knock-out = removing a gene. Knock-in = inserting a gene. Gene can be manipulated at specific developmental points using an inducible Cre-lox system with an antibiotic-controlled promoter (e.g., to study a gene whose deletion causes an embryonic lethal).
\end{itemize}
\end{frame}

%% page: 331
\begin{frame}
 \frametitle{Biochemistry 2}
RNAi–
\end{frame}

%% page: 332
\begin{frame}
 \frametitle{}
\begin{itemize}
\item dsRNA is synthesized that is complementary to the mRNA sequence of interest. When transfected into human cells, the dsRNA separates and promotes degradation of the target mRNA, knocking down gene expression.
\end{itemize}
\end{frame}

%% page: 333
\begin{frame}
 \frametitle{Biochemistry 2}
karyotyping
\end{frame}

%% page: 334
\begin{frame}
 \frametitle{}
\begin{itemize}
\item a process in which metaphase chromosomes are stained, ordered, and numbered according to morphology, size, arm-length ratio and banding pattern. Can be performed on a sample of blood, bone marrow, amniotic fluid, or placental tissue. Used to diagnose chromosomal imbalance (e. g. autosomal trisomies, microdeletions, sex chromosome disorders.) 
\end{itemize}
\end{frame}

%% page: 335
\begin{frame}
 \frametitle{Biochemistry 2}
Codominance
\end{frame}

%% page: 336
\begin{frame}
 \frametitle{}
\begin{itemize}
\item Neither of 2 alleles is dominant (e.g., blood groups).
\end{itemize}
\end{frame}

%% page: 337
\begin{frame}
 \frametitle{Biochemistry 2}
Variable expression
\end{frame}

%% page: 338
\begin{frame}
 \frametitle{}
\begin{itemize}
\item Nature and severity of the phenotype varies from 1 individual to another. \\ 2 patients from neurofibromatosis may have varying disease severity. 
\end{itemize}
\end{frame}

%% page: 339
\begin{frame}
 \frametitle{Biochemistry 2}
Incomplete penetrance 
\end{frame}

%% page: 340
\begin{frame}
 \frametitle{}
\begin{itemize}
\item Not all individuals with a mutant genotype show the mutant phenotype.
\end{itemize}
\end{frame}

%% page: 341
\begin{frame}
 \frametitle{Biochemistry 2}
Pleiotropy
\end{frame}

%% page: 342
\begin{frame}
 \frametitle{}
\begin{itemize}
\item 1 gene has $>$ 1 effect on an individual’s phenotype. \\ PKU causes many seemingly unrelated symptoms ranging from mental retardation to hair/skin changes. 
\end{itemize}
\end{frame}

%% page: 343
\begin{frame}
 \frametitle{Biochemistry 2}
Imprinting
\end{frame}

%% page: 344
\begin{frame}
 \frametitle{}
\begin{itemize}
\item Differences in phenotype depend on whether the mutation is of maternal or paternal \\ origin (e.g., AngelMan’s syndrome [Maternal], Prader-Willi syndrome [Paternal]).
\end{itemize}
\end{frame}

%% page: 345
\begin{frame}
 \frametitle{Biochemistry 2}
Anticipation
\end{frame}

%% page: 346
\begin{frame}
 \frametitle{}
\begin{itemize}
\item Severity of disease worsens or age of onset of disease is earlier in succeeding generations \\ (e.g., Huntington’s disease).
\end{itemize}
\end{frame}

%% page: 347
\begin{frame}
 \frametitle{Biochemistry 2}
Loss of heterozygosity
\end{frame}

%% page: 348
\begin{frame}
 \frametitle{}
\begin{itemize}
\item If a patient inherits or develops a mutation in a tumor suppressor gene, the complementary \\ allele must be deleted/mutated before cancer develops. This is not true of oncogenes. \\ retinoblastoma 
\end{itemize}
\end{frame}

%% page: 349
\begin{frame}
 \frametitle{Biochemistry 2}
Dominant negative mutation
\end{frame}

%% page: 350
\begin{frame}
 \frametitle{}
\begin{itemize}
\item Exerts a dominant effect. A heterozygote produces a nonfunctional altered protein that \\ also prevents the normal gene product from functioning. \\ mutation of Tx factor in its allosteric site. Nonfunctioning mutant can still bind DNA, preventing wild-type Tx factor from binding. 
\end{itemize}
\end{frame}

%% page: 351
\begin{frame}
 \frametitle{Biochemistry 2}
Linkage disequilibrium
\end{frame}

%% page: 352
\begin{frame}
 \frametitle{}
\begin{itemize}
\item Tendency for certain alleles at 2 linked loci to occur together more often than \\ expected by chance. Measured in a population, not in a family, and often varies in \\ different populations.
\end{itemize}
\end{frame}

%% page: 353
\begin{frame}
 \frametitle{Biochemistry 2}
Mosaicism
\end{frame}

%% page: 354
\begin{frame}
 \frametitle{}
\begin{itemize}
\item Occurs when cells in the body have different genetic makeup (e.g., lyonization–– \\ random X inactivation in females). \\ can be a germ-line mosaic, which may produce disease that is not carried by parent's somatic cells. 
\end{itemize}
\end{frame}

%% page: 355
\begin{frame}
 \frametitle{Biochemistry 2}
Locus heterogeneity
\end{frame}

%% page: 356
\begin{frame}
 \frametitle{}
\begin{itemize}
\item Mutations at different loci can produce the same phenotype (e.g., albinism). \\ marfan's syndrome, MEN 2B, and homocystinuria, all cause marfanoid habitus. 
\end{itemize}
\end{frame}

%% page: 357
\begin{frame}
 \frametitle{Biochemistry 2}
Heteroplasmy
\end{frame}

%% page: 358
\begin{frame}
 \frametitle{}
\begin{itemize}
\item Presence of both normal and mutated mtDNA, resulting in variable expression in \\ mitochondrial inherited diseases.
\end{itemize}
\end{frame}

%% page: 359
\begin{frame}
 \frametitle{Biochemistry 2}
Uniparental disomy
\end{frame}

%% page: 360
\begin{frame}
 \frametitle{}
\begin{itemize}
\item Offspring receives 2 copies of a chromosome from 1 parent and no copies from the \\ other parent
\end{itemize}
\end{frame}

%% page: 361
\begin{frame}
 \frametitle{Biochemistry 2}
Hardy-Weinberg law assumes:
\end{frame}

%% page: 362
\begin{frame}
 \frametitle{}
\begin{itemize}
\item There is no mutation occurring at the locus There is no selection for any of the genotypes at the locus
Mating is completely random Thereisnomigrationintoor out of the population
being considered
\end{itemize}
\end{frame}

%% page: 363
\begin{frame}
 \frametitle{Biochemistry 2}
Hardy-Weinberg population genetics
\end{frame}

%% page: 364
\begin{frame}
 \frametitle{}
\begin{itemize}
\item Disease prevalence: p2 + 2pq + q2 = 1 \\ Allele prevalence:	p + q = 1 p and q are separate alleles; 2pq = heterozygote \\ prevalence. The prevalence of an X-linked recessive disease in males = q and in females = q2
\end{itemize}
\end{frame}

%% page: 365
\begin{frame}
 \frametitle{Biochemistry 2}
Imprinting
\end{frame}

%% page: 366
\begin{frame}
 \frametitle{}
\begin{itemize}
\item At a single locus, only 1 allele is active; the other is inactive (imprinted/inactivated by methylation). Deletion of the active allele ${\rightarrow}$ disease. \\ Can also occur as a result of uniparental disomy. \\ Chromosome 15. 
\end{itemize}
\end{frame}

%% page: 367
\begin{frame}
 \frametitle{Biochemistry 2}
Prader-Willi syndrome
\end{frame}

%% page: 368
\begin{frame}
 \frametitle{}
\begin{itemize}
\item Deletion of normally active paternal allele. \\ Mental retardation, obesity, hypogonadism, hypotonia.
\end{itemize}
\end{frame}

%% page: 369
\begin{frame}
 \frametitle{Biochemistry 2}
Angelman’s syndrome
\end{frame}

%% page: 370
\begin{frame}
 \frametitle{}
\begin{itemize}
\item Deletion of normally active maternal allele. \\ Mental retardation, seizures, \\ ataxia, inappropriate laughter (happy puppet).
\end{itemize}
\end{frame}

%% page: 371
\begin{frame}
 \frametitle{Biochemistry 2}
Autosomal dominant
\end{frame}

%% page: 372
\begin{frame}
 \frametitle{}
\begin{itemize}
\item Often due to defects in structural genes. Many generations, both male and female, affected. \\ Often pleiotropic and, in many cases, present clinically after puberty. Family history crucial to diagnosis.
\end{itemize}
\end{frame}

%% page: 373
\begin{frame}
 \frametitle{Biochemistry 2}
Autosomal recessive
\end{frame}

%% page: 374
\begin{frame}
 \frametitle{}
\begin{itemize}
\item 25\% of offspring from 2 carrier parents are affected. Often due to enzyme deficiencies. Usually seen in only 1 generation. \\ Commonly more severe than dominant disorders; patients often present in childhood.
\end{itemize}
\end{frame}

%% page: 375
\begin{frame}
 \frametitle{Biochemistry 2}
Hypophosphatemic rickets.
\end{frame}

%% page: 376
\begin{frame}
 \frametitle{}
\begin{itemize}
\item X-linked dominant \\ formed known as vitamin D-resistant rickets. Inherited disorder resulting in increase phosphate wasting at proximal tubule. Results in rickets-like presentation. 
\end{itemize}
\end{frame}

%% page: 377
\begin{frame}
 \frametitle{Biochemistry 2}
Mitochondrial inheritance
\end{frame}

%% page: 378
\begin{frame}
 \frametitle{}
\begin{itemize}
\item Transmitted only through mother. All offspring of affected females may show signs of disease. \\ Variable expression in population due to heteroplasmy.
\end{itemize}
\end{frame}

%% page: 379
\begin{frame}
 \frametitle{Biochemistry 2}
Leber’s hereditary optic neuropathy
\end{frame}

%% page: 380
\begin{frame}
 \frametitle{}
\begin{itemize}
\item mitochondrial myopathies. \\ degeneration of retinal ganglion cells and axons. Leads to acute loss of central vision. 
\end{itemize}
\end{frame}

%% page: 381
\begin{frame}
 \frametitle{Biochemistry 2}
Achondroplasia
\end{frame}

%% page: 382
\begin{frame}
 \frametitle{}
\begin{itemize}
\item Autosomal-dominant cell-signaling defect of fibroblast growth factor (FGF) receptor 3. Results in dwarfism; short limbs, but head and trunk are normal size. Associated with advanced paternal age.
\end{itemize}
\end{frame}

%% page: 383
\begin{frame}
 \frametitle{Biochemistry 2}
Adult polycystic kidney disease
\end{frame}

%% page: 384
\begin{frame}
 \frametitle{}
\begin{itemize}
\item Always bilateral, massive enlargement of kidneys due to multiple large cysts. Patients present with pain, hematuria, hypertension, progressive renal failure. 90\% of cases are due to mutation in APKD1 (chromosome 16). Associated with polycystic liver disease, berry aneurysms, mitral valve prolapse. Juvenile form is recessive.
\end{itemize}
\end{frame}

%% page: 385
\begin{frame}
 \frametitle{Biochemistry 2}
Familial adenomatous polyposis
\end{frame}

%% page: 386
\begin{frame}
 \frametitle{}
\begin{itemize}
\item Colon becomes covered with adenomatous polyps after puberty. Progresses to colon cancer unless resected. Deletion on chromosome 5 (APC gene); 5 letters in “polyp.”
\end{itemize}
\end{frame}

%% page: 387
\begin{frame}
 \frametitle{Biochemistry 2}
Familial hypercholesterolemia (hyperlipidemia type IIA)
\end{frame}

%% page: 388
\begin{frame}
 \frametitle{}
\begin{itemize}
\item Elevated LDL owing to defective or absent LDL receptor. Heterozygotes (1:500) have cholesterol ${\simeq}$ 300 mg/dL. Homozygotes (very rare) have cholesterol ${\simeq}$ 700+ mg/dL, severe atherosclerotic disease early in life, and tendon xanthomas (classically in the Achilles tendon); MI may develop before age 20.
\end{itemize}
\end{frame}

%% page: 389
\begin{frame}
 \frametitle{Biochemistry 2}
hereditary hemorrhagic telangiectasia ( Osler-weber-Rendu syndrome) 
\end{frame}

%% page: 390
\begin{frame}
 \frametitle{}
\begin{itemize}
\item  inherited disorder of blood vessels. Findings: telangiectasia, recurrent epistaxis, skin discolorations, arteriovenous malformations ( AVMs).
\end{itemize}
\end{frame}

%% page: 391
\begin{frame}
 \frametitle{Biochemistry 2}
Hereditary spherocytosis
\end{frame}

%% page: 392
\begin{frame}
 \frametitle{}
\begin{itemize}
\item Spheroid erythrocytes; hemolytic anemia; ${\uparrow}$ MCHC. Splenectomy is curative.
\end{itemize}
\end{frame}

%% page: 393
\begin{frame}
 \frametitle{Biochemistry 2}
Huntington’s disease
\end{frame}

%% page: 394
\begin{frame}
 \frametitle{}
\begin{itemize}
\item Findings: depression, progressive dementia, choreiform movements, caudate atrophy, and ${\downarrow}$ levels of GABA and ACh in the brain. Symptoms manifest in affected individuals between the ages of 20 and 50. Gene located on chromosome 4; triplet repeat disorder. “Hunting 4 food.”
\end{itemize}
\end{frame}

%% page: 395
\begin{frame}
 \frametitle{Biochemistry 2}
Marfan’s syndrome
\end{frame}

%% page: 396
\begin{frame}
 \frametitle{}
\begin{itemize}
\item Fibrillin gene mutation ${\rightarrow}$ connective tissue disorders. Skeletal abnormalities––tall with long extremities, pectus excavatum, hyperextensive \\ joints, and long, tapering fingers and toes (arachnodactyly; see Figure 110). Cardiovascular––cystic medial necrosis of aorta ${\rightarrow}$ aortic incompetence and \\ dissecting aortic aneurysms. Floppy mitral valve. Ocular––subluxation of lenses.
\end{itemize}
\end{frame}

%% page: 397
\begin{frame}
 \frametitle{Biochemistry 2}
multiple endocrine neoplasias
\end{frame}

%% page: 398
\begin{frame}
 \frametitle{}
\begin{itemize}
\item  several distinct syndromes ( 1, 2A, 2B) characterized by familial tumors of endocrine glands, including pancreas, parathyroids,pituitary, thyroid, and adrenal medulla. MEN2A, 2B are associated with ret gene. 
\end{itemize}
\end{frame}

%% page: 399
\begin{frame}
 \frametitle{Biochemistry 2}
Neurofibromatosis type 1 (von Recklinghausen’s disease)
\end{frame}

%% page: 400
\begin{frame}
 \frametitle{}
\begin{itemize}
\item Findings: café-au-lait spots, neural tumors, Lisch nodules (pigmented iris hamartomas). Also marked by skeletal disorders (e.g., scoliosis), optic pathway gliomas, pheochromocytoma, and ${\uparrow}$ tumor susceptibility. On long arm of chromosome 17; 17 letters in von Recklinghausen.
\end{itemize}
\end{frame}

%% page: 401
\begin{frame}
 \frametitle{Biochemistry 2}
Neurofibromatosis type 2
\end{frame}

%% page: 402
\begin{frame}
 \frametitle{}
\begin{itemize}
\item Bilateral acoustic neuroma, juvenile cataracts. NF2 gene on chromosome 22; type 2 = 22.
\end{itemize}
\end{frame}

%% page: 403
\begin{frame}
 \frametitle{Biochemistry 2}
Tuberous sclerosis
\end{frame}

%% page: 404
\begin{frame}
 \frametitle{}
\begin{itemize}
\item Findings: facial lesions (adenoma sebaceum), hypopigmented “ash leaf spots” on skin, cortical and retinal hamartomas, seizures, mental retardation, renal cysts and renal angiomyolipomas, cardiac rhabdomyomas, ${\uparrow}$ incidence of astrocytomas. Incomplete penetrance, variable presentation.
\end{itemize}
\end{frame}

%% page: 405
\begin{frame}
 \frametitle{Biochemistry 2}
von Hippel–Lindau disease
\end{frame}

%% page: 406
\begin{frame}
 \frametitle{}
\begin{itemize}
\item Findings: hemangioblastomas of retina/cerebellum/medulla; about half of affected individuals develop multiple bilateral renal cell carcinomas and other tumors. Associated with deletion of VHL gene (tumor suppressor) on chromosome 3 (3p). Von Hippel–Lindau = 3 words for chromosome 3. \\ constitutive expression of HIF ( transcription factor) and activation of angiogenic growth factor. 
\end{itemize}
\end{frame}

%% page: 407
\begin{frame}
 \frametitle{Biochemistry 2}
Cystic fibrosis
\end{frame}

%% page: 408
\begin{frame}
 \frametitle{}
\begin{itemize}
\item Autosomal-recessive defect in CFTR gene on chromosome 7, commonly deletion of Phe 508. CFTRchannelactivelysecretesCl– inlungsand GI tract and actively reabsorbs Cl– from sweat. \\ Defective Cl- channel ${\rightarrow}$ secretion of abnormally thick mucus that plugs lungs, pancreas, and liver ${\rightarrow}$ recurrent pulmonary infections (Pseudomonas species and S. aureus), \\ chronic bronchitis, bronchiectasis, pancreatic insufficiency (malabsorption and steatorrhea), meconium ileus in newborns. ${\uparrow}$ concentration of Cl- ions in sweat test is diagnostic. \\ Treatment: N-acetylcysteine to loosen mucous plugs.
\end{itemize}
\end{frame}

%% page: 409
\begin{frame}
 \frametitle{Biochemistry 2}
Duchenne’s (X-linked)
\end{frame}

%% page: 410
\begin{frame}
 \frametitle{}
\begin{itemize}
\item Frame-shift mutation ${\rightarrow}$ deletion of dystrophin gene ${\rightarrow}$ accelerated muscle breakdown. Onset before 5 years of age. Weakness begins in pelvic girdle muscles and progresses superiorly. Pseudohypertrophy of calf muscles due to fibrofatty replacement of muscle; cardiac myopathy. The use of Gowers’ maneuver, requiring assistance of the upper extremities to stand up, is characteristic (indicates proximal lower limb weakness). \\ Diagnose muscular dystrophies by ${\uparrow}$ CPK and muscle biopsy.
\end{itemize}
\end{frame}

%% page: 411
\begin{frame}
 \frametitle{Biochemistry 2}
Try (trinucleotide) hunting for my fried eggs (X).
\end{frame}

%% page: 412
\begin{frame}
 \frametitle{}
\begin{itemize}
\item huntington's disease= CAG. Myotonic dystrophy= CTG, Fragile X syndrome= CGG, Friedreich's ataxia= GAA 
\end{itemize}
\end{frame}

%% page: 413
\begin{frame}
 \frametitle{Biochemistry 2}
Down syndrome 
\end{frame}

%% page: 414
\begin{frame}
 \frametitle{}
\begin{itemize}
\item flate facies, congenital heart disease ( most commonly septum-primum-type ASD). Associated with increase risk of ALL and Alzheimer's disease ( $>$ 35 years old.)  \\ 95\% of cases due to meiotic nondisjunction of homologous chromosomes ( associated with advanced maternal age). 4\% of cases due to robertsonian translocation. 1\% of cases due to down mosaicism ( no maternal association). 
\end{itemize}
\end{frame}

%% page: 415
\begin{frame}
 \frametitle{Biochemistry 2}
pregnancy screen of down syndrome
\end{frame}

%% page: 416
\begin{frame}
 \frametitle{}
\begin{itemize}
\item decrease a-fetoprotein, decrease estriol, increase b-hCG, increase inhibin A. 
\end{itemize}
\end{frame}

%% page: 417
\begin{frame}
 \frametitle{Biochemistry 2}
edwards syndrome
\end{frame}

%% page: 418
\begin{frame}
 \frametitle{}
\begin{itemize}
\item congenital heart disease, death usually occurs within 1 year of birth. Most common trisomy resulting in live birth after down syndrome. 
\end{itemize}
\end{frame}

%% page: 419
\begin{frame}
 \frametitle{Biochemistry 2}
patau's syn 
\end{frame}

%% page: 420
\begin{frame}
 \frametitle{}
\begin{itemize}
\item congenital heart disease, death usually occurs within 1 year of birth. Holoprosencephaly. 
\end{itemize}
\end{frame}

%% page: 421
\begin{frame}
 \frametitle{Biochemistry 2}
robertsonian translocation 
\end{frame}

%% page: 422
\begin{frame}
 \frametitle{}
\begin{itemize}
\item nonrecciprocal chromosomal translocation.  \\ unbalanced translocations  \\ balanced translocations. 
\end{itemize}
\end{frame}

%% page: 423
\begin{frame}
 \frametitle{Biochemistry 2}
nonreciprocal chromosal translocation 
\end{frame}

%% page: 424
\begin{frame}
 \frametitle{}
\begin{itemize}
\item  commonly involves chromosome pairs 13, 14, 15, 21 and 22 \\ long arems of 2 acrocentric chromosomes ( chromosomes with centromeres near their ends) fuse at the centromere and the 2 short arms are lost. 
\end{itemize}
\end{frame}

%% page: 425
\begin{frame}
 \frametitle{Biochemistry 2}
balanced translocations 
\end{frame}

%% page: 426
\begin{frame}
 \frametitle{}
\begin{itemize}
\item normally do not cause any abnomal phenotype. 
\end{itemize}
\end{frame}

%% page: 427
\begin{frame}
 \frametitle{Biochemistry 2}
unbalanced translocations 
\end{frame}

%% page: 428
\begin{frame}
 \frametitle{}
\begin{itemize}
\item result in miscarriage, still birth, and chromosomal imbalance ( e. g. down syndrome, patau's syndrome) 
\end{itemize}
\end{frame}

%% page: 429
\begin{frame}
 \frametitle{Biochemistry 2}
Chromosomal inversions
\end{frame}

%% page: 430
\begin{frame}
 \frametitle{}
\begin{itemize}
\item chromosome rearrangement in which a segment of a chromosome is reversed end to end. May result in decrease fertility. 
\end{itemize}
\end{frame}

%% page: 431
\begin{frame}
 \frametitle{Biochemistry 2}
Pericentric
\end{frame}

%% page: 432
\begin{frame}
 \frametitle{}
\begin{itemize}
\item Involves centromere; proceeds through meiosis.
\end{itemize}
\end{frame}

%% page: 433
\begin{frame}
 \frametitle{Biochemistry 2}
Paracentric 
\end{frame}

%% page: 434
\begin{frame}
 \frametitle{}
\begin{itemize}
\item Does not involve centromere; does not proceed through meiosis.
\end{itemize}
\end{frame}

%% page: 435
\begin{frame}
 \frametitle{Biochemistry 2}
williams syndrome
\end{frame}

%% page: 436
\begin{frame}
 \frametitle{}
\begin{itemize}
\item congenital microdeletion of long arm of chromosome 7 ( deleted region includes elastin gene). Findings= distinctive “elfin” facies, mental retardation, hypercalcemia ( increase sensitivity to vitamin D), well-developed verbal skills, extreme friendliness with strangers, cardiovascular problems. 
\end{itemize}
\end{frame}

%% page: 437
\begin{frame}
 \frametitle{Biochemistry 2}
rich in Rough endoplamic reticulum 
\end{frame}

%% page: 438
\begin{frame}
 \frametitle{}
\begin{itemize}
\item Mucus-secreting goblet cells of the small intestine and antibody-secreting plasma cells. 
\end{itemize}
\end{frame}

%% page: 439
\begin{frame}
 \frametitle{Biochemistry 2}
Cell cycle phases
\end{frame}

%% page: 440
\begin{frame}
 \frametitle{}
\begin{itemize}
\item Checkpoints control transitions between phases; regulated by cyclins, CDKs, and tumor suppressors \\ Mitosis (shortest phase): prophase-metaphase- anaphase-telophase. G1 and G0 are of variable duration. \\ G stands for Gap or Growth; S for Synthesis.
\end{itemize}
\end{frame}

%% page: 441
\begin{frame}
 \frametitle{Biochemistry 2}
CDKs–
\end{frame}

%% page: 442
\begin{frame}
 \frametitle{}
\begin{itemize}
\item Cyclin-dependent kinases, constitutive and inactive. 
\end{itemize}
\end{frame}

%% page: 443
\begin{frame}
 \frametitle{Biochemistry 2}
Cyclins––
\end{frame}

%% page: 444
\begin{frame}
 \frametitle{}
\begin{itemize}
\item regulatory proteins that control cell cycle events.  \\ phase specific, activate CDKs.
\end{itemize}
\end{frame}

%% page: 445
\begin{frame}
 \frametitle{Biochemistry 2}
Cyclin-CDK complexes
\end{frame}

%% page: 446
\begin{frame}
 \frametitle{}
\begin{itemize}
\item must be both activated \\ and inactivated for cell cycle to progress.
\end{itemize}
\end{frame}

%% page: 447
\begin{frame}
 \frametitle{Biochemistry 2}
Rb and p53 tumor suppressors 
\end{frame}

%% page: 448
\begin{frame}
 \frametitle{}
\begin{itemize}
\item normally
inhibit G1-to-S progression; mutations in these genes result in unrestrained growth.
\end{itemize}
\end{frame}

%% page: 449
\begin{frame}
 \frametitle{Biochemistry 2}
Permanent cells
\end{frame}

%% page: 450
\begin{frame}
 \frametitle{}
\begin{itemize}
\item Remain in G0, regenerate from stem cells. \\ Neurons, skeletal and cardiac muscle cells, and RBCs remain in G0.
\end{itemize}
\end{frame}

%% page: 451
\begin{frame}
 \frametitle{Biochemistry 2}
Stable (quiescent) cells
\end{frame}

%% page: 452
\begin{frame}
 \frametitle{}
\begin{itemize}
\item Enter G1 from G0 when stimulated. \\ Hepatocytes, lymphocytes.
\end{itemize}
\end{frame}

%% page: 453
\begin{frame}
 \frametitle{Biochemistry 2}
Labile cells
\end{frame}

%% page: 454
\begin{frame}
 \frametitle{}
\begin{itemize}
\item Never go to G0, divide rapidly with a short G1. \\ Bone marrow, gut epithelium, skin, hair follicles.
\end{itemize}
\end{frame}

%% page: 455
\begin{frame}
 \frametitle{Biochemistry 2}
Functions of Golgi apparatus
\end{frame}

%% page: 456
\begin{frame}
 \frametitle{}
\begin{itemize}
\item . Distribution center of proteins and lipids from ER to the plasma membrane, lysosomes, and secretory vesicles \\ 2. Modifies N-oligosaccharides on asparagine 3. Adds O-oligosaccharides to serine and threonine \\ residues 4. Addition of mannose-6-phosphate to specific \\ lysosomal proteins, which targets the protein \\ to the lysosome 5. Proteoglycan assembly from proteoglycan core \\ proteins 6. Sulfation of sugars in proteoglycans and of \\ selected tyrosine on proteins
\end{itemize}
\end{frame}

%% page: 457
\begin{frame}
 \frametitle{Biochemistry 2}
Vesicular trafficking proteins:
\end{frame}

%% page: 458
\begin{frame}
 \frametitle{}
\begin{itemize}
\item COPI: retrograde, \\ Golgi ${\rightarrow}$ ER. COPII: anterograde, RER ${\rightarrow}$ cis-Golgi. \\ Clathrin: trans-Golgi ${\rightarrow}$ lysosomes, plasma membrane ${\rightarrow}$ endosomes (receptor-mediated endocytosis).
\end{itemize}
\end{frame}

%% page: 459
\begin{frame}
 \frametitle{Biochemistry 2}
present of chediak higashi syndrome 
\end{frame}

%% page: 460
\begin{frame}
 \frametitle{}
\begin{itemize}
\item  recurrent pyogenic infections, partial albinism, and peripheral neuropathy.
\end{itemize}
\end{frame}

%% page: 461
\begin{frame}
 \frametitle{Biochemistry 2}
Microtubule
\end{frame}

%% page: 462
\begin{frame}
 \frametitle{}
\begin{itemize}
\item Cylindrical structure composed of a helical array of polymerized dimers of ${\alpha}$- and ${\beta}$-tubulin. Each dimer has 2 GTP bound. Incorporated into flagella, cilia, mitotic spindles. Grows slowly, collapses quickly. Microtubules are also involved in slow axoplasmic transport in neurons.
\end{itemize}
\end{frame}

%% page: 463
\begin{frame}
 \frametitle{Biochemistry 2}
Kartagener’s syndrome
\end{frame}

%% page: 464
\begin{frame}
 \frametitle{}
\begin{itemize}
\item Results in male and female infertility (sperm immotile), bronchiectasis, and recurrent sinusitis (bacteria and particles not pushed out); associated with situs inversus.
\end{itemize}
\end{frame}

%% page: 465
\begin{frame}
 \frametitle{Biochemistry 2}
Plasma membrane composition
\end{frame}

%% page: 466
\begin{frame}
 \frametitle{}
\begin{itemize}
\item Asymmetric fluid bilayer. Contains cholesterol (~50\%), phospholipids (~50\%), sphingolipids, glycolipids, \\ and proteins. High cholesterol or long saturated fatty acid content ${\rightarrow}$ ${\uparrow}$ melting temperature, ${\downarrow}$ fluidity.
\end{itemize}
\end{frame}

%% page: 467
\begin{frame}
 \frametitle{Biochemistry 2}
Ouabain
\end{frame}

%% page: 468
\begin{frame}
 \frametitle{}
\begin{itemize}
\item inhibits Na+-K+ ATPase  by binding to K+ site.
\end{itemize}
\end{frame}

%% page: 469
\begin{frame}
 \frametitle{Biochemistry 2}
Cardiac glycosides
\end{frame}

%% page: 470
\begin{frame}
 \frametitle{}
\begin{itemize}
\item digoxin and digitoxin from foxglove) also inhibit the Na+-K+ ATPase, causing ${\uparrow}$ cardiac contractility. \\ indirect inhibition of Na/Ca exchange. Increase [Ca] inside of the cardiac muslce cell
\end{itemize}
\end{frame}

%% page: 471
\begin{frame}
 \frametitle{Biochemistry 2}
Be (So Totally) Cool, Read Books.
\end{frame}

%% page: 472
\begin{frame}
 \frametitle{}
\begin{itemize}
\item Type I: BONE. \\ Type II: carTWOlage. \\ Type III (Reticulin) \\ Type IV: Under the floor (basement membrane).
\end{itemize}
\end{frame}

%% page: 473
\begin{frame}
 \frametitle{Biochemistry 2}
Collagen synthesis and structure
\end{frame}

%% page: 474
\begin{frame}
 \frametitle{}
\begin{itemize}
\item Inside fibroblasts \\ 1. Synthesis (RER) \\ 2. Hydroxylation (ER) \\ 3. Glycosylation (ER) \\ 4. Exocytosis \\ Outside fibroblasts \\ 5. Proteolytic processing \\ 6. Cross-linking
\end{itemize}
\end{frame}

%% page: 475
\begin{frame}
 \frametitle{Biochemistry 2}
Synthesis (RER)
\end{frame}

%% page: 476
\begin{frame}
 \frametitle{}
\begin{itemize}
\item Translation of collagen ${\alpha}$ chains (preprocollagen)— usually Gly-X-Y polypeptide (X and Y are proline, hydroxyproline, or hydroxylysine).
\end{itemize}
\end{frame}

%% page: 477
\begin{frame}
 \frametitle{Biochemistry 2}
Hydroxylation (ER)
\end{frame}

%% page: 478
\begin{frame}
 \frametitle{}
\begin{itemize}
\item Hydroxylation of specific proline and lysine residues (requires vitamin C).
\end{itemize}
\end{frame}

%% page: 479
\begin{frame}
 \frametitle{Biochemistry 2}
Glycosylation (ER)
\end{frame}

%% page: 480
\begin{frame}
 \frametitle{}
\begin{itemize}
\item Glycosylation of pro-${\alpha}$-chain lysine residues and formation of procollagen (triple helix of 3 collagen ${\alpha}$ chains).
\end{itemize}
\end{frame}

%% page: 481
\begin{frame}
 \frametitle{Biochemistry 2}
Exocytosis
\end{frame}

%% page: 482
\begin{frame}
 \frametitle{}
\begin{itemize}
\item Exocytosis of procollagen into extracellular space.
\end{itemize}
\end{frame}

%% page: 483
\begin{frame}
 \frametitle{Biochemistry 2}
Proteolytic processing
\end{frame}

%% page: 484
\begin{frame}
 \frametitle{}
\begin{itemize}
\item Cleavage of terminal regions of procollagen transforms it into insoluble tropocollagen.
\end{itemize}
\end{frame}

%% page: 485
\begin{frame}
 \frametitle{Biochemistry 2}
Cross-linking
\end{frame}

%% page: 486
\begin{frame}
 \frametitle{}
\begin{itemize}
\item Reinforcement of many staggered tropocollagen molecules by covalent lysine-hydroxylysine cross-linkage (by lysyl oxidase) to make collagen fibrils.
\end{itemize}
\end{frame}

%% page: 487
\begin{frame}
 \frametitle{Biochemistry 2}
Ehlers-Danlos syndrome
\end{frame}

%% page: 488
\begin{frame}
 \frametitle{}
\begin{itemize}
\item Faulty collagen synthesis causing: 1. Hyperextensible skin 2. Tendency to bleed (easy bruising) 3. Hypermobile joints
\end{itemize}
\end{frame}

%% page: 489
\begin{frame}
 \frametitle{Biochemistry 2}
Osteogenesis imperfecta
\end{frame}

%% page: 490
\begin{frame}
 \frametitle{}
\begin{itemize}
\item Most common form is autosomal-dominant with abnormal collagen type I. \\ 1. Multiple fractures occurring with minimal trauma (brittle bone disease), which may occur during the birth process \\ 2. Blue sclerae due to the translucency of the connective tissue over the choroid \\ 3. Hearing loss (abnormal middle ear bones) 4. Dental imperfections due to lack of dentin \\ May be confused with child abuse. \\ Type II is fatal in utero or in the neonatal period. \\ Incidence is 1:10,000.
\end{itemize}
\end{frame}

%% page: 491
\begin{frame}
 \frametitle{Biochemistry 2}
alport's syndrome 
\end{frame}

%% page: 492
\begin{frame}
 \frametitle{}
\begin{itemize}
\item due to a variety of gene defects resulting in abnormal type IV collagen. Most common form is X-linked recessive. Characterized by progressive hereditary nephritis and deafness. May be associated with ocular disturbances. Typer IV collagen is an important structural component of the basement membrane of the kidney, ears, and eyes. 
\end{itemize}
\end{frame}


\end{document}

