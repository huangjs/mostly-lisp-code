\documentclass[11pt]{beamer}
\beamersetaveragebackground{green!80!gray}
\usepackage[T1]{fontenc}
\usepackage{fixltx2e}
\usepackage{graphicx}
\usepackage{longtable}
\usepackage{float}
\usepackage{wrapfig}
\usepackage{soul}
\usepackage{textcomp}
\usepackage{marvosym}
\usepackage{latexsym}
\usepackage{amssymb}
\usepackage{hyperref}
\tolerance=1000
\usepackage{amsmath}
\usepackage{wasysym}
\usepackage{color}
\usepackage{xcolor}
\usepackage{upquote}
\usepackage{listings}
\usepackage{tikz}
\usepackage{fancyvrb}
\usepackage{fontspec}
\usepackage{xunicode}
\usepackage{xltxtra}
\usepackage{xeCJK}

\title{First Aid 2010 (Pharm)}
\author{huangjs}

\begin{document}


%% page: 1
\begin{frame}
 \frametitle{}
sonic hedgehog gene 
\end{frame}

%% page: 2
\begin{frame}
 \frametitle{}
\begin{itemize}
\item{produced at base of limbs in zone of polarizing activity, involved in patterning along anterior posterior axis }
\end{itemize}
\end{frame}

%% page: 3
\begin{frame}
 \frametitle{}
wnt-7 gene
\end{frame}

%% page: 4
\begin{frame}
 \frametitle{}
\begin{itemize}
\item{Produced at apical ectodermal ridge (thickened ectoderm at distal end of each developing limb). Necessary for proper organization along dorsal-ventral axis.}
\end{itemize}
\end{frame}

%% page: 5
\begin{frame}
 \frametitle{}
FGF gene
\end{frame}

%% page: 6
\begin{frame}
 \frametitle{}
\begin{itemize}
\item{Produced at apical ectodermal ridge. Stimulates mitosis of underlying mesoderm. providing for lengthening of limbs.}
\end{itemize}
\end{frame}

%% page: 7
\begin{frame}
 \frametitle{}
Homeobox gene
\end{frame}

%% page: 8
\begin{frame}
 \frametitle{}
\begin{itemize}
\item{Involved in segmental organization}
\end{itemize}
\end{frame}

%% page: 9
\begin{frame}
 \frametitle{}
Day 0
\end{frame}

%% page: 10
\begin{frame}
 \frametitle{}
\begin{itemize}
\item{Fertilization by sperm forming \\ zygote, initiating embryogenesis}
\end{itemize}
\end{frame}

%% page: 11
\begin{frame}
 \frametitle{}
Within week 1
\end{frame}

%% page: 12
\begin{frame}
 \frametitle{}
\begin{itemize}
\item{Implantation (as a blastocyst). \\ hGG secretion begins }
\end{itemize}
\end{frame}

%% page: 13
\begin{frame}
 \frametitle{}
Within week 2
\end{frame}

%% page: 14
\begin{frame}
 \frametitle{}
\begin{itemize}
\item{Bilaminar disk (epiblast, hypoblast}
\end{itemize}
\end{frame}

%% page: 15
\begin{frame}
 \frametitle{}
Within week 3
\end{frame}

%% page: 16
\begin{frame}
 \frametitle{}
\begin{itemize}
\item{Gastrulation. \\ Primitive streak,	notochord, \\ and neural plate begin to form.}
\end{itemize}
\end{frame}

%% page: 17
\begin{frame}
 \frametitle{}
Weeks 3–8
\end{frame}

%% page: 18
\begin{frame}
 \frametitle{}
\begin{itemize}
\item{Organogenesis. Extremely susceptible to teratogens. \\ Neural tube formed by neuroectoderm and closed by week 4.}
\end{itemize}
\end{frame}

%% page: 19
\begin{frame}
 \frametitle{}
Week 4
\end{frame}

%% page: 20
\begin{frame}
 \frametitle{}
\begin{itemize}
\item{Heart begins to beat. \\ Upper and lower limb buds begin to form.}
\end{itemize}
\end{frame}

%% page: 21
\begin{frame}
 \frametitle{}
Week 8
\end{frame}

%% page: 22
\begin{frame}
 \frametitle{}
\begin{itemize}
\item{Fetal movement; fetus looks like a baby.}
\end{itemize}
\end{frame}

%% page: 23
\begin{frame}
 \frametitle{}
Week 10
\end{frame}

%% page: 24
\begin{frame}
 \frametitle{}
\begin{itemize}
\item{Genitalia have male/female characteristics.}
\end{itemize}
\end{frame}

%% page: 25
\begin{frame}
 \frametitle{}
Alar plate (dorsal)
\end{frame}

%% page: 26
\begin{frame}
 \frametitle{}
\begin{itemize}
\item{Sensory, same orientation as spinal cord }
\end{itemize}
\end{frame}

%% page: 27
\begin{frame}
 \frametitle{}
Basal plate (ventral)
\end{frame}

%% page: 28
\begin{frame}
 \frametitle{}
\begin{itemize}
\item{Motor}
\end{itemize}
\end{frame}

%% page: 29
\begin{frame}
 \frametitle{}
Rule of 2’s for 2nd week
\end{frame}

%% page: 30
\begin{frame}
 \frametitle{}
\begin{itemize}
\item{2 germ layers (bilaminar disk): epiblast, hypoblast. 2 cavities: amniotic cavity, yolk sac. 2 components to placenta: cytotrophoblast, \\ syncytiotrophoblast.}
\end{itemize}
\end{frame}

%% page: 31
\begin{frame}
 \frametitle{}
Rule of 3’s for 3rd week
\end{frame}

%% page: 32
\begin{frame}
 \frametitle{}
\begin{itemize}
\item{3 germ layers (gastrula): ectoderm, mesoderm, \\ endoderm.}
\end{itemize}
\end{frame}

%% page: 33
\begin{frame}
 \frametitle{}
Rule of 4’s for 4th week
\end{frame}

%% page: 34
\begin{frame}
 \frametitle{}
\begin{itemize}
\item{4 heart chambers. 4 limb buds grow.}
\end{itemize}
\end{frame}

%% page: 35
\begin{frame}
 \frametitle{}
Surface ectoderm
\end{frame}

%% page: 36
\begin{frame}
 \frametitle{}
\begin{itemize}
\item{Adenohypophysis ( from Rathke's pouch); lens of eye; epithelial linings of skin, ear, eye, and nose; epidermis. \\ retina, and olfactory epithelium, salivary, sweat, and mammary glands. }
\end{itemize}
\end{frame}

%% page: 37
\begin{frame}
 \frametitle{}
Neuroectoderm
\end{frame}

%% page: 38
\begin{frame}
 \frametitle{}
\begin{itemize}
\item{Neurohypophysis, CNS neurons, oligodendrocytes, astrocytes, ependymal cells, pineal gland. \\ retina, spinal cord , think CNS and brain. }
\end{itemize}
\end{frame}

%% page: 39
\begin{frame}
 \frametitle{}
Neural crest
\end{frame}

%% page: 40
\begin{frame}
 \frametitle{}
\begin{itemize}
\item{ANS, dorsal root ganglia, melanocytes, chromaffin cells of adrenal medulla, \\ enterochromaffin cells, pia and arachnoid, celiac ganglion, Schwann cells, odontoblasts, \\ parafollicular (C) cells of thyroid, laryngeal cartilage, bones of the skull. \\ cranial nerves, celiac ganglion \\ Neural crest-think PNS and non-neural structures nearby. \\ Odonto = teeth, Think Crest toothpaste. }
\end{itemize}
\end{frame}

%% page: 41
\begin{frame}
 \frametitle{}
cranipharyngioma 
\end{frame}

%% page: 42
\begin{frame}
 \frametitle{}
\begin{itemize}
\item{benign Rathke's pouch tumor with chlosterol crystals, calcifications. }
\end{itemize}
\end{frame}

%% page: 43
\begin{frame}
 \frametitle{}
Endoderm
\end{frame}

%% page: 44
\begin{frame}
 \frametitle{}
\begin{itemize}
\item{Gut tube epithelium and derivatives (e.g., lungs, liver, pancreas, thymus, parathyroid, thyroid \\ follicular cells).}
\end{itemize}
\end{frame}

%% page: 45
\begin{frame}
 \frametitle{}
Mesoderm
\end{frame}

%% page: 46
\begin{frame}
 \frametitle{}
\begin{itemize}
\item{Muscle, bone, connective tissue, serous linings of body cavities (e.g.. peritoneum).  spleen (derived from foregut mesentery), cardiovascular strutures, lymphatics, blood, urogenital structures. Kidneys. adrenal cortex. \\ Nolochord induces ectoderm to form neuroectoderm (neural plate). Its postnatal derivative is Ihe nucleus pulposus of the inlerverlebral disk.}
\end{itemize}
\end{frame}

%% page: 47
\begin{frame}
 \frametitle{}
Teratogens
\end{frame}

%% page: 48
\begin{frame}
 \frametitle{}
\begin{itemize}
\item{Most susceptible in 3rd–8th weeks (organogenesis) of pregnancy. \\ Before week 3: all-or-none effects. After week 8: growth and function affected. }
\end{itemize}
\end{frame}

%% page: 49
\begin{frame}
 \frametitle{}
AGE inhibitors
\end{frame}

%% page: 50
\begin{frame}
 \frametitle{}
\begin{itemize}
\item{Renal damage}
\end{itemize}
\end{frame}

%% page: 51
\begin{frame}
 \frametitle{}
Alcohol
\end{frame}

%% page: 52
\begin{frame}
 \frametitle{}
\begin{itemize}
\item{Leading cause of birth defects and mental retardation; fetal alcohol syndrome}
\end{itemize}
\end{frame}

%% page: 53
\begin{frame}
 \frametitle{}
Alkylating agents
\end{frame}

%% page: 54
\begin{frame}
 \frametitle{}
\begin{itemize}
\item{Absence of digits, multiple anomalies}
\end{itemize}
\end{frame}

%% page: 55
\begin{frame}
 \frametitle{}
Aminoglycosides
\end{frame}

%% page: 56
\begin{frame}
 \frametitle{}
\begin{itemize}
\item{CNVIII toxicitv}
\end{itemize}
\end{frame}

%% page: 57
\begin{frame}
 \frametitle{}
CoCaine
\end{frame}

%% page: 58
\begin{frame}
 \frametitle{}
\begin{itemize}
\item{Abnormal fetal development  \\ and fetal addiction; placental \\ abruption }
\end{itemize}
\end{frame}

%% page: 59
\begin{frame}
 \frametitle{}
Diethylstilbestrol (DES)
\end{frame}

%% page: 60
\begin{frame}
 \frametitle{}
\begin{itemize}
\item{Vaginal clear cell \\ adenocarcinoma}
\end{itemize}
\end{frame}

%% page: 61
\begin{frame}
 \frametitle{}
Iodide
\end{frame}

%% page: 62
\begin{frame}
 \frametitle{}
\begin{itemize}
\item{Congenital goiter or \\ hypothyroidism}
\end{itemize}
\end{frame}

%% page: 63
\begin{frame}
 \frametitle{}
Lithium
\end{frame}

%% page: 64
\begin{frame}
 \frametitle{}
\begin{itemize}
\item{Ebstein's anomaly ( strialized right ventricle) }
\end{itemize}
\end{frame}

%% page: 65
\begin{frame}
 \frametitle{}
Maternal diabetes
\end{frame}

%% page: 66
\begin{frame}
 \frametitle{}
\begin{itemize}
\item{Caudal regression syndrome \\ (anal atresia to sirenomelia)}
\end{itemize}
\end{frame}

%% page: 67
\begin{frame}
 \frametitle{}
Smoking (nicotine. GO)
\end{frame}

%% page: 68
\begin{frame}
 \frametitle{}
\begin{itemize}
\item{Preterm labor, placental \\ problems, IUGR, ADHD}
\end{itemize}
\end{frame}

%% page: 69
\begin{frame}
 \frametitle{}
Tetracyclines
\end{frame}

%% page: 70
\begin{frame}
 \frametitle{}
\begin{itemize}
\item{Discolored teeth}
\end{itemize}
\end{frame}

%% page: 71
\begin{frame}
 \frametitle{}
Thalidomide
\end{frame}

%% page: 72
\begin{frame}
 \frametitle{}
\begin{itemize}
\item{Limb defects ("flipper" limbs)}
\end{itemize}
\end{frame}

%% page: 73
\begin{frame}
 \frametitle{}
Valproate
\end{frame}

%% page: 74
\begin{frame}
 \frametitle{}
\begin{itemize}
\item{inhibition of intestinal folate absorption}
\end{itemize}
\end{frame}

%% page: 75
\begin{frame}
 \frametitle{}
Vitamin A (excess)
\end{frame}

%% page: 76
\begin{frame}
 \frametitle{}
\begin{itemize}
\item{Extremely high risk for \\ spontaneous abortions and birth defects (cleft palate. cardiac abnormalities) }
\end{itemize}
\end{frame}

%% page: 77
\begin{frame}
 \frametitle{}
Warfarin
\end{frame}

%% page: 78
\begin{frame}
 \frametitle{}
\begin{itemize}
\item{Bone deformities, fetal hemorrhage, abortion}
\end{itemize}
\end{frame}

%% page: 79
\begin{frame}
 \frametitle{}
X-rays, anticonvulsants
\end{frame}

%% page: 80
\begin{frame}
 \frametitle{}
\begin{itemize}
\item{Multiple anomalies }
\end{itemize}
\end{frame}

%% page: 81
\begin{frame}
 \frametitle{}
Fetal alcohol syndrome
\end{frame}

%% page: 82
\begin{frame}
 \frametitle{}
\begin{itemize}
\item{Leading cause of congenital malformations in the United States. Newborns of mothers who consumed significant amounts of alcohol during pregnancy have an ${\uparrow}$ incidence of congenital abnormalities, including pre- and postnatal developmental retardation, microcephaly, facial abnormalities, limb dislocation, and heart and lung fistulas. Mechanism may include inhibition of cell migration.}
\end{itemize}
\end{frame}

%% page: 83
\begin{frame}
 \frametitle{}
Monozygotic
\end{frame}

%% page: 84
\begin{frame}
 \frametitle{}
\begin{itemize}
\item{1 zygote splits evenly to develop 2 amniotic sacs with a single common chorion and placenta. \\ Conjoined twins have 1 chorion, 1 amniotic sac.}
\end{itemize}
\end{frame}

%% page: 85
\begin{frame}
 \frametitle{}
Dizygotic (fraternal) or monozygotic
\end{frame}

%% page: 86
\begin{frame}
 \frametitle{}
\begin{itemize}
\item{Monozygotes that split early develop 2 placentas (separate/ttised), chorions, and amniotic sacs. \\ Dizygotes develop individual placentas, chorions, and amniotic sacs.}
\end{itemize}
\end{frame}

%% page: 87
\begin{frame}
 \frametitle{}
Fetal component
\end{frame}

%% page: 88
\begin{frame}
 \frametitle{}
\begin{itemize}
\item{Cytotrophoblast composes the inner layer of chorionic villi.Cyto makes Cells.  Syncytiotrophoblast is the \\ outer layer and secretes hCG ( structurally similar to LH ; stimulates  corpus luteum to secrete
progesterone during first trimester).}
\end{itemize}
\end{frame}

%% page: 89
\begin{frame}
 \frametitle{}
Maternal component
\end{frame}

%% page: 90
\begin{frame}
 \frametitle{}
\begin{itemize}
\item{Decidua basalis derived from the endometrium. \\ maternal blood in lacunae. }
\end{itemize}
\end{frame}

%% page: 91
\begin{frame}
 \frametitle{}
2 umbilical arteries,
\end{frame}

%% page: 92
\begin{frame}
 \frametitle{}
\begin{itemize}
\item{return deoxygenated blood from fetal internal iliac arteries to placenta,}
\end{itemize}
\end{frame}

%% page: 93
\begin{frame}
 \frametitle{}
1 umbilical vein,
\end{frame}

%% page: 94
\begin{frame}
 \frametitle{}
\begin{itemize}
\item{supplies oxygenated blood from the placenta to the fetus.}
\end{itemize}
\end{frame}

%% page: 95
\begin{frame}
 \frametitle{}
Single umbilical artery
\end{frame}

%% page: 96
\begin{frame}
 \frametitle{}
\begin{itemize}
\item{associated with congenital and chromosomal anomalies. \\ Umbilical arteries and veins are derived from allantois.}
\end{itemize}
\end{frame}

%% page: 97
\begin{frame}
 \frametitle{}
3rd week—yolk sac forms allantois,
\end{frame}

%% page: 98
\begin{frame}
 \frametitle{}
\begin{itemize}
\item{Allantois extends into urogenital sinus. Allantois becomes urachus. a duct between bladder and yolk sac.}
\end{itemize}
\end{frame}

%% page: 99
\begin{frame}
 \frametitle{}
Failure of urachus to obliterate:
\end{frame}

%% page: 100
\begin{frame}
 \frametitle{}
\begin{itemize}
\item{1. Patent urachus —urine discharge \\ from umbilicus 2. Vesicourachal diverticulum — \\ outpouching of bladder}
\end{itemize}
\end{frame}

%% page: 101
\begin{frame}
 \frametitle{}
7th week —obliteration of vitelline duct 
\end{frame}

%% page: 102
\begin{frame}
 \frametitle{}
\begin{itemize}
\item{ omphalomesenteric duct. Connects yolk sac to midgut lumen. }
\end{itemize}
\end{frame}

%% page: 103
\begin{frame}
 \frametitle{}
Vitelline fistula 
\end{frame}

%% page: 104
\begin{frame}
 \frametitle{}
\begin{itemize}
\item{failure of duct to close —$>$ meconium discharge from umbilicus. Examples: Meckel's diverticulum—partial closure. with patent portion attached to ileum. May have ectopic gastric mucosa —$>$ melena and RUQpaiu.}
\end{itemize}
\end{frame}

%% page: 105
\begin{frame}
 \frametitle{}
Truncus arteriosus (TA)
\end{frame}

%% page: 106
\begin{frame}
 \frametitle{}
\begin{itemize}
\item{Ascending aorta and pulmonary trunk}
\end{itemize}
\end{frame}

%% page: 107
\begin{frame}
 \frametitle{}
Bulbus cordis
\end{frame}

%% page: 108
\begin{frame}
 \frametitle{}
\begin{itemize}
\item{Right ventricle and smooth parts \\ (outflow tract) of left and right ventricle}
\end{itemize}
\end{frame}

%% page: 109
\begin{frame}
 \frametitle{}
Primitive ventricle
\end{frame}

%% page: 110
\begin{frame}
 \frametitle{}
\begin{itemize}
\item{Portion of the left ventricle}
\end{itemize}
\end{frame}

%% page: 111
\begin{frame}
 \frametitle{}
Primitive atria
\end{frame}

%% page: 112
\begin{frame}
 \frametitle{}
\begin{itemize}
\item{Trabeculatcd left and \\ right atrium}
\end{itemize}
\end{frame}

%% page: 113
\begin{frame}
 \frametitle{}
left horn of sinus venosus (SV)
\end{frame}

%% page: 114
\begin{frame}
 \frametitle{}
\begin{itemize}
\item{coronary sinus}
\end{itemize}
\end{frame}

%% page: 115
\begin{frame}
 \frametitle{}
Right horn of SV
\end{frame}

%% page: 116
\begin{frame}
 \frametitle{}
\begin{itemize}
\item{Smooth part of right atrium}
\end{itemize}
\end{frame}

%% page: 117
\begin{frame}
 \frametitle{}
Right common cardinal vein and right anterior cardinal vein 
\end{frame}

%% page: 118
\begin{frame}
 \frametitle{}
\begin{itemize}
\item{SVC}
\end{itemize}
\end{frame}

%% page: 119
\begin{frame}
 \frametitle{}
Truncus arteriosus
\end{frame}

%% page: 120
\begin{frame}
 \frametitle{}
\begin{itemize}
\item{Neural crest migration —$>$ divide trunk into 2 arteries via fusion and twisting of truncal and bulbar ridges —$>$ ascending aorta and pulmonary trunk. \\ Palhology—transposition of great vessels + tetralogy of Fallot.}
\end{itemize}
\end{frame}

%% page: 121
\begin{frame}
 \frametitle{}
Interventricular septum development
\end{frame}

%% page: 122
\begin{frame}
 \frametitle{}
\begin{itemize}
\item{1. Muscular ventricular septum forms. Opening is called the interventricular foramen. \\ 2. The aorticopulmonary septum divides the TA into the aortic and pulmonary trunks. 3. The aorticopulmonary septum meets and fuses with the muscular ventricular septum \\ to form the membranous interventricular septum, closing the interventricular foramen.}
\end{itemize}
\end{frame}

%% page: 123
\begin{frame}
 \frametitle{}
Interatrial septum development
\end{frame}

%% page: 124
\begin{frame}
 \frametitle{}
\begin{itemize}
\item{1. Foramen primum narrows as the septum primum grows toward the endocardial cushions. \\ 2. Perforations in the septum primum form the foramen secundum as the foramen primum disappears. \\ 3. The foramen secundum maintains the right-to-left shunt as the septum secundum begins to grow. \\ 4. The septum secundum contains a permanent opening called the foramen ovale. 5. The foramen secundum enlarges and the upper part of the septum primum ovale.  \\ degenerates. 6. The remaining portion of the septum primum is now called the valve of the foramen}
\end{itemize}
\end{frame}

%% page: 125
\begin{frame}
 \frametitle{}
Fetal erythropoiesis occurs in:
\end{frame}

%% page: 126
\begin{frame}
 \frametitle{}
\begin{itemize}
\item{1. Yolk sac (3–8 wk) 2. Liver (6–30 wk) 3. Spleen (9–28 wk) \\ 4. Bone marrow (28 wk onward) \\ Young Liver Synthesizes Blood. \\ Fetal hemoglobin = ${\alpha}$2${\gamma}$2. Adult hemoglobin = ${\alpha}$2${\beta}$2.}
\end{itemize}
\end{frame}

%% page: 127
\begin{frame}
 \frametitle{}
Blood in umbilical vein
\end{frame}

%% page: 128
\begin{frame}
 \frametitle{}
\begin{itemize}
\item{ 80\% saturated with O2.}
\end{itemize}
\end{frame}

%% page: 129
\begin{frame}
 \frametitle{}
Umbilical arteries
\end{frame}

%% page: 130
\begin{frame}
 \frametitle{}
\begin{itemize}
\item{low O2 saturation.}
\end{itemize}
\end{frame}

%% page: 131
\begin{frame}
 \frametitle{}
3 important shunts in fetas 
\end{frame}

%% page: 132
\begin{frame}
 \frametitle{}
\begin{itemize}
\item{1. Blood entering the fetus \\ through the umbilical vein is conducted via the ductus venosus into the IVC to bypass the hepatic circulation.  \\ 2. Most oxygenated blood reaching the heart via the IVC is diverted through the foramen ovale and pumped out the aorta to the head and body. \\ 3. Deoxygenated blood from the SVC is expelled into the pulmonary artery and ductus arteriosus to the lower body of the fetus.}
\end{itemize}
\end{frame}

%% page: 133
\begin{frame}
 \frametitle{}
infant take a breath 
\end{frame}

%% page: 134
\begin{frame}
 \frametitle{}
\begin{itemize}
\item{${\downarrow}$ resistance in pulmonary vasculature causes ${\uparrow}$ left atrial pressure vs. right atrial pressure; foramen ovale closes (now called fossa ovalis); ${\uparrow}$ in O2 leads to \\ ${\downarrow}$ in prostaglandins, causing \\ closure of ductus arteriosus.}
\end{itemize}
\end{frame}

%% page: 135
\begin{frame}
 \frametitle{}
Indomethacin
\end{frame}

%% page: 136
\begin{frame}
 \frametitle{}
\begin{itemize}
\item{close PDA.}
\end{itemize}
\end{frame}

%% page: 137
\begin{frame}
 \frametitle{}
Prostaglandins
\end{frame}

%% page: 138
\begin{frame}
 \frametitle{}
\begin{itemize}
\item{keep PDA open.}
\end{itemize}
\end{frame}

%% page: 139
\begin{frame}
 \frametitle{}
Umbilical vein–
\end{frame}

%% page: 140
\begin{frame}
 \frametitle{}
\begin{itemize}
\item{ligamentum teres hepatis}
\end{itemize}
\end{frame}

%% page: 141
\begin{frame}
 \frametitle{}
UmbiLical arteries
\end{frame}

%% page: 142
\begin{frame}
 \frametitle{}
\begin{itemize}
\item{mediaL umbilical ligaments}
\end{itemize}
\end{frame}

%% page: 143
\begin{frame}
 \frametitle{}
Ductus arteriosus
\end{frame}

%% page: 144
\begin{frame}
 \frametitle{}
\begin{itemize}
\item{ligamentum arteriosum}
\end{itemize}
\end{frame}

%% page: 145
\begin{frame}
 \frametitle{}
Ductus venosus
\end{frame}

%% page: 146
\begin{frame}
 \frametitle{}
\begin{itemize}
\item{ligamentum venosum}
\end{itemize}
\end{frame}

%% page: 147
\begin{frame}
 \frametitle{}
Foramen ovale
\end{frame}

%% page: 148
\begin{frame}
 \frametitle{}
\begin{itemize}
\item{fossa ovalis}
\end{itemize}
\end{frame}

%% page: 149
\begin{frame}
 \frametitle{}
AllaNtois––urachus––
\end{frame}

%% page: 150
\begin{frame}
 \frametitle{}
\begin{itemize}
\item{mediaN umbilical ligament}
\end{itemize}
\end{frame}

%% page: 151
\begin{frame}
 \frametitle{}
Notochord–
\end{frame}

%% page: 152
\begin{frame}
 \frametitle{}
\begin{itemize}
\item{nucleus pulposus of intervertebral disk}
\end{itemize}
\end{frame}

%% page: 153
\begin{frame}
 \frametitle{}
1st aortic arch derivatives 
\end{frame}

%% page: 154
\begin{frame}
 \frametitle{}
\begin{itemize}
\item{part of Maxillary artery ( branch of external carotid). 1st arch is Maximal. }
\end{itemize}
\end{frame}

%% page: 155
\begin{frame}
 \frametitle{}
2nd aortic arch derivatives 
\end{frame}

%% page: 156
\begin{frame}
 \frametitle{}
\begin{itemize}
\item{Stapedial artery and hyoid artery.}
\end{itemize}
\end{frame}

%% page: 157
\begin{frame}
 \frametitle{}
3rd–
\end{frame}

%% page: 158
\begin{frame}
 \frametitle{}
\begin{itemize}
\item{common Carotid artery and proximal part of \\ internal carotid artery.C is 3rd letter of alphabet.}
\end{itemize}
\end{frame}

%% page: 159
\begin{frame}
 \frametitle{}
4th–
\end{frame}

%% page: 160
\begin{frame}
 \frametitle{}
\begin{itemize}
\item{on left, aortic arch; on right, proximal part of \\ right subclavian artery.4th arch (4 limbs) = systemic.}
\end{itemize}
\end{frame}

%% page: 161
\begin{frame}
 \frametitle{}
6th––
\end{frame}

%% page: 162
\begin{frame}
 \frametitle{}
\begin{itemize}
\item{proximal part of pulmonary arteries and (on \\ left only) ductus arteriosus.6th arch = pulmonary and the
pulmonary-to-systemic shunt (ductus arteriosus).}
\end{itemize}
\end{frame}

%% page: 163
\begin{frame}
 \frametitle{}
presencephalon 
\end{frame}

%% page: 164
\begin{frame}
 \frametitle{}
\begin{itemize}
\item{Telencephalon, diecephalon. }
\end{itemize}
\end{frame}

%% page: 165
\begin{frame}
 \frametitle{}
cerebral hemipheres 
\end{frame}

%% page: 166
\begin{frame}
 \frametitle{}
\begin{itemize}
\item{lateral ventricles}
\end{itemize}
\end{frame}

%% page: 167
\begin{frame}
 \frametitle{}
Thalamis etc 
\end{frame}

%% page: 168
\begin{frame}
 \frametitle{}
\begin{itemize}
\item{3rd ventricle}
\end{itemize}
\end{frame}

%% page: 169
\begin{frame}
 \frametitle{}
Maternal component
\end{frame}

%% page: 170
\begin{frame}
 \frametitle{}
\begin{itemize}
\item{medibrain}
\end{itemize}
\end{frame}

%% page: 171
\begin{frame}
 \frametitle{}
rhombencephalon 
\end{frame}

%% page: 172
\begin{frame}
 \frametitle{}
\begin{itemize}
\item{metencephalon and myelencephalon }
\end{itemize}
\end{frame}

%% page: 173
\begin{frame}
 \frametitle{}
metencephalon 
\end{frame}

%% page: 174
\begin{frame}
 \frametitle{}
\begin{itemize}
\item{pons and cerebellum }
\end{itemize}
\end{frame}

%% page: 175
\begin{frame}
 \frametitle{}
myelencephalon
\end{frame}

%% page: 176
\begin{frame}
 \frametitle{}
\begin{itemize}
\item{medulla }
\end{itemize}
\end{frame}

%% page: 177
\begin{frame}
 \frametitle{}
Neuroporcs fail lo fuse (4th week)
\end{frame}

%% page: 178
\begin{frame}
 \frametitle{}
\begin{itemize}
\item{persistent connection between amniotic cavity and spinal canal. Associated with low folic acid intake during pregnancy. Elevated \\ a-fetoprotein (AFP) in amniotic fluid and maternal serum. increase AFP + acetylcholinesterase in CSF. }
\end{itemize}
\end{frame}

%% page: 179
\begin{frame}
 \frametitle{}
Spina bifida occulta—
\end{frame}

%% page: 180
\begin{frame}
 \frametitle{}
\begin{itemize}
\item{failure of bony spinal canal to close, but no structural herniation. Usuallv seen at lower vertebral levels. Dura is intact}
\end{itemize}
\end{frame}

%% page: 181
\begin{frame}
 \frametitle{}
Meningocele 
\end{frame}

%% page: 182
\begin{frame}
 \frametitle{}
\begin{itemize}
\item{meninges herniate through spinal canal defect }
\end{itemize}
\end{frame}

%% page: 183
\begin{frame}
 \frametitle{}
Myelomeningocele 
\end{frame}

%% page: 184
\begin{frame}
 \frametitle{}
\begin{itemize}
\item{meninges and spinal cord herniate through spinal canal defect }
\end{itemize}
\end{frame}

%% page: 185
\begin{frame}
 \frametitle{}
Anencephaly
\end{frame}

%% page: 186
\begin{frame}
 \frametitle{}
\begin{itemize}
\item{Malformation of anterior end of neural lube; no brain and calvarium, elevated AFP, polyhydramnios (no swallowing center in brain).}
\end{itemize}
\end{frame}

%% page: 187
\begin{frame}
 \frametitle{}
Holoprosencephaly 
\end{frame}

%% page: 188
\begin{frame}
 \frametitle{}
\begin{itemize}
\item{decrease separation of hemispheres across midline: results in cyclopia; associated with Patau's syndrome, severe fetal alcohol syndrome, and cleft lip/palate.}
\end{itemize}
\end{frame}

%% page: 189
\begin{frame}
 \frametitle{}
Arnold-chiari type II—
\end{frame}

%% page: 190
\begin{frame}
 \frametitle{}
\begin{itemize}
\item{cerebellar tonsillar herniation through foramen magnum with aqueductal stenosis and hydrocephaly.often presents with syringomyelia, thoracolumbar myelomeningocel. }
\end{itemize}
\end{frame}

%% page: 191
\begin{frame}
 \frametitle{}
Dandy-Walker
\end{frame}

%% page: 192
\begin{frame}
 \frametitle{}
\begin{itemize}
\item{large posterior fossa; absent cerebellar vermis with cystic enlargement of 4th ventricle. Can lead lo hydrocephalus and spina bifida}
\end{itemize}
\end{frame}

%% page: 193
\begin{frame}
 \frametitle{}
Syringomyelia
\end{frame}

%% page: 194
\begin{frame}
 \frametitle{}
\begin{itemize}
\item{enlargement of the central canal of spinal cord. Crossing fibers of spinothalanmic tract are typically damaged first. "Cape-like," bilateral loss of pain and temperature sensation in upper extremities with preservation of touch sensation. \\ Syrinx (Greek) = tube, as in syringe. \\ Often presents in patients with Arnold-Chiari II malformation. \\ Most common at C8-T1.}
\end{itemize}
\end{frame}

%% page: 195
\begin{frame}
 \frametitle{}
Branchial apparatus
\end{frame}

%% page: 196
\begin{frame}
 \frametitle{}
\begin{itemize}
\item{Composed of branchial clefts, arches, and pouches. \\  Also called pharyngeal apparatus.}
\end{itemize}
\end{frame}

%% page: 197
\begin{frame}
 \frametitle{}
Branchial clefts
\end{frame}

%% page: 198
\begin{frame}
 \frametitle{}
\begin{itemize}
\item{derived from ectoderm, also called branchial grooves.  \\ Also called branchial grooves.}
\end{itemize}
\end{frame}

%% page: 199
\begin{frame}
 \frametitle{}
Branchial arches
\end{frame}

%% page: 200
\begin{frame}
 \frametitle{}
\begin{itemize}
\item{derived from mesoderm ( muscle and arteries) and neural crests( bones and cartilage). }
\end{itemize}
\end{frame}

%% page: 201
\begin{frame}
 \frametitle{}
Branchial pouches
\end{frame}

%% page: 202
\begin{frame}
 \frametitle{}
\begin{itemize}
\item{derived from endoderm}
\end{itemize}
\end{frame}

%% page: 203
\begin{frame}
 \frametitle{}
CAP covers outside from inside
\end{frame}

%% page: 204
\begin{frame}
 \frametitle{}
\begin{itemize}
\item{Clefts = ectoderm, Arches = mesoderm, Pouches = endoderm}
\end{itemize}
\end{frame}

%% page: 205
\begin{frame}
 \frametitle{}
Branchial arch 1 derivatives
\end{frame}

%% page: 206
\begin{frame}
 \frametitle{}
\begin{itemize}
\item{Meckel’s cartilage: Mandible, Malleus, incus, sphenoMandibular ligament. Muscles: Muscles of Mastication (temporalis, Masseter, lateral and Medial pterygoids), \\ Mylohyoid, anterior belly of digastric, tensor tympani, tensor veli palatini, anterior \\ 2/3 of tongue. Nerve: CN V2 and CN V3.}
\end{itemize}
\end{frame}

%% page: 207
\begin{frame}
 \frametitle{}
Branchial arch 2 derivatives
\end{frame}

%% page: 208
\begin{frame}
 \frametitle{}
\begin{itemize}
\item{Reichert’s cartilage: Stapes, Styloid process, lesser horn of hyoid, Stylohyoid ligament. Muscles: muscles of facial expression, Stapedius, Stylohyoid, posterior belly of digastric. Nerve: CN VII.}
\end{itemize}
\end{frame}

%% page: 209
\begin{frame}
 \frametitle{}
Branchial arch 3 derivatives
\end{frame}

%% page: 210
\begin{frame}
 \frametitle{}
\begin{itemize}
\item{Cartilage: greater horn of hyoid. Muscle: stylopharyngeus. Nerve: CN IX.}
\end{itemize}
\end{frame}

%% page: 211
\begin{frame}
 \frametitle{}
Branchial arches 4–6 derivatives
\end{frame}

%% page: 212
\begin{frame}
 \frametitle{}
\begin{itemize}
\item{Cartilages: thyroid, cricoid, arytenoids, corniculate, cuneiform. \\ Muscles (4th arch): most pharyngeal constrictors, cricothyroid, levator veli palatini. \\ Muscles (6th arch): all intrinsic muscles of larynx \\ except cricothyroid. \\ Nerve: 4th arch––CN X (superior laryngeal branch); 6th arch––CN X (recurrent laryngeal branch).}
\end{itemize}
\end{frame}

%% page: 213
\begin{frame}
 \frametitle{}
Arch 1 derivatives supplied
\end{frame}

%% page: 214
\begin{frame}
 \frametitle{}
\begin{itemize}
\item{ CN V2 and V3.}
\end{itemize}
\end{frame}

%% page: 215
\begin{frame}
 \frametitle{}
Arch 2 derivatives
\end{frame}

%% page: 216
\begin{frame}
 \frametitle{}
\begin{itemize}
\item{CN VII.}
\end{itemize}
\end{frame}

%% page: 217
\begin{frame}
 \frametitle{}
Arch 3
\end{frame}

%% page: 218
\begin{frame}
 \frametitle{}
\begin{itemize}
\item{CN IX. }
\end{itemize}
\end{frame}

%% page: 219
\begin{frame}
 \frametitle{}
Arch 4 and 6 derivatives
\end{frame}

%% page: 220
\begin{frame}
 \frametitle{}
\begin{itemize}
\item{by CN X.}
\end{itemize}
\end{frame}

%% page: 221
\begin{frame}
 \frametitle{}
Branchial arch innervation
\end{frame}

%% page: 222
\begin{frame}
 \frametitle{}
\begin{itemize}
\item{These Cns are the only ones with both \\ .sensory and motor components (except \\ V2. which is sensory only). Think of arches in terms of actions-chewing {1) \\ facial expression (2l. Stvlopharyngeus (3), swallowing (4), speaking (6}i}
\end{itemize}
\end{frame}

%% page: 223
\begin{frame}
 \frametitle{}
1st cleft develops into
\end{frame}

%% page: 224
\begin{frame}
 \frametitle{}
\begin{itemize}
\item{external auditory meatus.}
\end{itemize}
\end{frame}

%% page: 225
\begin{frame}
 \frametitle{}
2nd through 4th clefts form 
\end{frame}

%% page: 226
\begin{frame}
 \frametitle{}
\begin{itemize}
\item{temporary cervical sinuses, which are obliterated by \\ proliferation of 2nd arch mesenchyme.}
\end{itemize}
\end{frame}

%% page: 227
\begin{frame}
 \frametitle{}
Persistent cervical sinus
\end{frame}

%% page: 228
\begin{frame}
 \frametitle{}
\begin{itemize}
\item{ branchial cleft cvst within lateral neck.}
\end{itemize}
\end{frame}

%% page: 229
\begin{frame}
 \frametitle{}
1st pouch develops into
\end{frame}

%% page: 230
\begin{frame}
 \frametitle{}
\begin{itemize}
\item{middle ear cavity, eustachian tube, mastoid air cells. \\ 1st pouch contributes to endoderm-lined structures of ear.}
\end{itemize}
\end{frame}

%% page: 231
\begin{frame}
 \frametitle{}
2nd pouch develops into
\end{frame}

%% page: 232
\begin{frame}
 \frametitle{}
\begin{itemize}
\item{epithelial lining of palatine tonsil.}
\end{itemize}
\end{frame}

%% page: 233
\begin{frame}
 \frametitle{}
3rd pouch (dorsal wings) develops into 
\end{frame}

%% page: 234
\begin{frame}
 \frametitle{}
\begin{itemize}
\item{inferior parathyroids.3rd pouch contributes to 3 structures (thymus, left and right inferior parathyroids).}
\end{itemize}
\end{frame}

%% page: 235
\begin{frame}
 \frametitle{}
3rd pouch (ventral wings) develops into
\end{frame}

%% page: 236
\begin{frame}
 \frametitle{}
\begin{itemize}
\item{thymus.}
\end{itemize}
\end{frame}

%% page: 237
\begin{frame}
 \frametitle{}
4th pouch (dorsal wings) develops into
\end{frame}

%% page: 238
\begin{frame}
 \frametitle{}
\begin{itemize}
\item{superior parathyroids}
\end{itemize}
\end{frame}

%% page: 239
\begin{frame}
 \frametitle{}
Aberrant development of 3rd and 4th pouches 
\end{frame}

%% page: 240
\begin{frame}
 \frametitle{}
\begin{itemize}
\item{DiGeorge syndrome ${\rightarrow}$ leads to T-cell deficiency (thymic aplasia) and hypocalcemia (failure of parathyroid development).}
\end{itemize}
\end{frame}

%% page: 241
\begin{frame}
 \frametitle{}
MEN 2A:
\end{frame}

%% page: 242
\begin{frame}
 \frametitle{}
\begin{itemize}
\item{mutation of germline RET(neural crest cells). — Adrenal medulla (pheochromocytoma). — Parathyroid (tumor): 3rd/4lh pharyngeal pouch. — Parafollicular cells (medullary thyroid cancer): 4th/5th pharyngeal pouch}
\end{itemize}
\end{frame}

%% page: 243
\begin{frame}
 \frametitle{}
Ist arch
\end{frame}

%% page: 244
\begin{frame}
 \frametitle{}
\begin{itemize}
\item{Malleus/incus \\ Tensor tyMpani (V3)}
\end{itemize}
\end{frame}

%% page: 245
\begin{frame}
 \frametitle{}
2nd arch
\end{frame}

%% page: 246
\begin{frame}
 \frametitle{}
\begin{itemize}
\item{Stapes \\ Stapedius (VII)}
\end{itemize}
\end{frame}

%% page: 247
\begin{frame}
 \frametitle{}
1st cleft
\end{frame}

%% page: 248
\begin{frame}
 \frametitle{}
\begin{itemize}
\item{[external auditory meatus}
\end{itemize}
\end{frame}

%% page: 249
\begin{frame}
 \frametitle{}
1st branchial membrane
\end{frame}

%% page: 250
\begin{frame}
 \frametitle{}
\begin{itemize}
\item{Tympanic membrane}
\end{itemize}
\end{frame}

%% page: 251
\begin{frame}
 \frametitle{}
1st pouch
\end{frame}

%% page: 252
\begin{frame}
 \frametitle{}
\begin{itemize}
\item{Eustachian tube, middle ear cavity. mastoid air cells}
\end{itemize}
\end{frame}

%% page: 253
\begin{frame}
 \frametitle{}
Tongue development
\end{frame}

%% page: 254
\begin{frame}
 \frametitle{}
\begin{itemize}
\item{1st branchial arch forms anlerior:A (thus sensation via CNV ,, taste via CNVII). \\ 3rd and 4th arches fonn posterior 'A (thus sensation and taste mainly via CN IX, extreme posterior via CN X). \\ Motor inuervalion is via CN XII. Muscles oflhe longuc are derived from occipital \\ invotomcs. \\ Taste-GNVII. IX,X (solitary nucleus).
Pain-CN V3, IX, X. Molor-CNXII.}
\end{itemize}
\end{frame}

%% page: 255
\begin{frame}
 \frametitle{}
Thyroid development
\end{frame}

%% page: 256
\begin{frame}
 \frametitle{}
\begin{itemize}
\item{Thyroid diverticulum arises from floor of primitive pharynx, descends into neck. Connected to tongue by thyroglossal duct, which normally disappears but may persist as pyramidal lobe of thyroid. Foramen cecum is normal remnant of thyroglossal duct. Most common ectopic thyroid tissue site is the tongue. \\ Thyroglossal duct cyst in midline neck and will move with swallowing (vs. persistent cervical sinus leading to branchial cyst in lateral neck).}
\end{itemize}
\end{frame}

%% page: 257
\begin{frame}
 \frametitle{}
Cleft lip
\end{frame}

%% page: 258
\begin{frame}
 \frametitle{}
\begin{itemize}
\item{failure of fusion of the maxillary and medial nasal processes (formation of 1$^{\circ}$ palate).}
\end{itemize}
\end{frame}

%% page: 259
\begin{frame}
 \frametitle{}
Cleft palate
\end{frame}

%% page: 260
\begin{frame}
 \frametitle{}
\begin{itemize}
\item{failure of fusion of the lateral palatine \\ processes, the nasal septum, and/or the median palatine process (formation of 2$^{\circ}$ palate).}
\end{itemize}
\end{frame}

%% page: 261
\begin{frame}
 \frametitle{}
Diaphragm embryology
\end{frame}

%% page: 262
\begin{frame}
 \frametitle{}
\begin{itemize}
\item{Diaphragm is derived from: 1. Septum transversum==central tendon  2. Pleuroperitoneal folds 3. Body wall \\ 4. Dorsal mesentery of esophagus = crura \\ Several Parts Build Diaphragm.}
\end{itemize}
\end{frame}

%% page: 263
\begin{frame}
 \frametitle{}
“C3, 4, 5 keeps the diaphragm alive.”
\end{frame}

%% page: 264
\begin{frame}
 \frametitle{}
\begin{itemize}
\item{Diaphragm descends during
development but maintains innervation from above C3–C5. \\ Abdominal contents may herniate into the thorax because of incomplete development (diaphragmatic hernia) ${\rightarrow}$ hypoplasia of thoracic organs due to space compression, scaphoid abdomen, cyanosis. }
\end{itemize}
\end{frame}

%% page: 265
\begin{frame}
 \frametitle{}
Foregut
\end{frame}

%% page: 266
\begin{frame}
 \frametitle{}
\begin{itemize}
\item{pharynx to duodenum}
\end{itemize}
\end{frame}

%% page: 267
\begin{frame}
 \frametitle{}
Midgut–
\end{frame}

%% page: 268
\begin{frame}
 \frametitle{}
\begin{itemize}
\item{duodenum to transverse colon}
\end{itemize}
\end{frame}

%% page: 269
\begin{frame}
 \frametitle{}
 Hindgut
\end{frame}

%% page: 270
\begin{frame}
 \frametitle{}
\begin{itemize}
\item{–distal transverse colon to rectum}
\end{itemize}
\end{frame}

%% page: 271
\begin{frame}
 \frametitle{}
Devdopmcnlal defects of anterior abdominal wall due to failure of 
\end{frame}

%% page: 272
\begin{frame}
 \frametitle{}
\begin{itemize}
\item{-Rostral fold closure: sternal defects —Lateral fold closure: omphalocele,
Gastroschisis —Caudal fold closure: bladder exstrophy}
\end{itemize}
\end{frame}

%% page: 273
\begin{frame}
 \frametitle{}
Duodenal atresia
\end{frame}

%% page: 274
\begin{frame}
 \frametitle{}
\begin{itemize}
\item{failure to recanalize (trisomy 21),}
\end{itemize}
\end{frame}

%% page: 275
\begin{frame}
 \frametitle{}
jejunal, ileal, colonic atresia
\end{frame}

%% page: 276
\begin{frame}
 \frametitle{}
\begin{itemize}
\item{due lo vascular \\ accident (apple peel atresia),}
\end{itemize}
\end{frame}

%% page: 277
\begin{frame}
 \frametitle{}
6th week—
\end{frame}

%% page: 278
\begin{frame}
 \frametitle{}
\begin{itemize}
\item{midgut herniates through umbilical ring - \\ rapid growth}
\end{itemize}
\end{frame}

%% page: 279
\begin{frame}
 \frametitle{}
10th week 
\end{frame}

%% page: 280
\begin{frame}
 \frametitle{}
\begin{itemize}
\item{return to abdominal cavity- + rotate \\ around SMA \\ Pathology—intestinal obstruction, twisting around \\ SMA (volvulus).}
\end{itemize}
\end{frame}

%% page: 281
\begin{frame}
 \frametitle{}
Gastroschisis
\end{frame}

%% page: 282
\begin{frame}
 \frametitle{}
\begin{itemize}
\item{–failure of lateral body folds to fuse ${\rightarrow}$ extrusion of abdominal contents through abdominal folds.}
\end{itemize}
\end{frame}

%% page: 283
\begin{frame}
 \frametitle{}
Omphalocele––
\end{frame}

%% page: 284
\begin{frame}
 \frametitle{}
\begin{itemize}
\item{persistence of herniation of abdominal contents into umbilical cord. \\ covered by peritoneum }
\end{itemize}
\end{frame}

%% page: 285
\begin{frame}
 \frametitle{}
Tracheoesophageal fistula
\end{frame}

%% page: 286
\begin{frame}
 \frametitle{}
\begin{itemize}
\item{Abnoimal connection between esophagus and trachea. Most common subtype is blind upper esophagus wilh lower esophagus connected to \\ trachea, Results in cyanosis, choking and vomiting with feeding, air bubble on CXR, and polyhydramnios.}
\end{itemize}
\end{frame}

%% page: 287
\begin{frame}
 \frametitle{}
Congenital pyloric stenosis
\end{frame}

%% page: 288
\begin{frame}
 \frametitle{}
\begin{itemize}
\item{Hypertrophy of the pylorus causes obstruction. Palpable "olive" mass in epigastric region and nonbilious projectile vomiting at = 2 weeks of age. Treatment is surgical incision. Occurs in 1/600 live births, often in 1st-born males.}
\end{itemize}
\end{frame}

%% page: 289
\begin{frame}
 \frametitle{}
Pancreas is derived from
\end{frame}

%% page: 290
\begin{frame}
 \frametitle{}
\begin{itemize}
\item{the foregut. Ventral pancreatic bud becomes pancreatic head, uncinate process (lower half of head), and main pancreatic duct. Dorsal pancreatic bud becomes everything else (body, tail, isthmus, and accessory pancreatic duct).}
\end{itemize}
\end{frame}

%% page: 291
\begin{frame}
 \frametitle{}
Annular pancreas
\end{frame}

%% page: 292
\begin{frame}
 \frametitle{}
\begin{itemize}
\item{ventral pancreatic bud abnormally encircles 2nd part of duodenum; forms a ring of pancreatic tissue that may cause duodenal narrowing.}
\end{itemize}
\end{frame}

%% page: 293
\begin{frame}
 \frametitle{}
Pancreas divisum
\end{frame}

%% page: 294
\begin{frame}
 \frametitle{}
\begin{itemize}
\item{ventral and dorsal parts fail to fuse at 8 weeks.}
\end{itemize}
\end{frame}

%% page: 295
\begin{frame}
 \frametitle{}
Spleen arises from
\end{frame}

%% page: 296
\begin{frame}
 \frametitle{}
\begin{itemize}
\item{arises from dorsal mesentery (hence is mesodermal) but is supplied by artery of \\ foregut (celiac artery).}
\end{itemize}
\end{frame}

%% page: 297
\begin{frame}
 \frametitle{}
Pronephros
\end{frame}

%% page: 298
\begin{frame}
 \frametitle{}
\begin{itemize}
\item{week 4; then degenerates}
\end{itemize}
\end{frame}

%% page: 299
\begin{frame}
 \frametitle{}
Mesonephros 
\end{frame}

%% page: 300
\begin{frame}
 \frametitle{}
\begin{itemize}
\item{functious as interim kidney for 1st trimester; later contributes to \\ male genital system}
\end{itemize}
\end{frame}

%% page: 301
\begin{frame}
 \frametitle{}
Metanephros
\end{frame}

%% page: 302
\begin{frame}
 \frametitle{}
\begin{itemize}
\item{permanent; beginnings first appear during 5th week of gestation; \\ nephrogenesis continues through 32-36 weeks of gestation}
\end{itemize}
\end{frame}

%% page: 303
\begin{frame}
 \frametitle{}
 Ureteric bud 
\end{frame}

%% page: 304
\begin{frame}
 \frametitle{}
\begin{itemize}
\item{derived from caudal end of mesonephros; gives rise to ureter, \\ pelvises, and, through branching, calyces and collecting ducts; fully canalized by 10th week}
\end{itemize}
\end{frame}

%% page: 305
\begin{frame}
 \frametitle{}
Metanephric mesenchyme
\end{frame}

%% page: 306
\begin{frame}
 \frametitle{}
\begin{itemize}
\item{ureteric bud interacts with this tissue; interaction induces differentiation and formation of glomerulus and renal tubules to distal convoluted tubule}
\end{itemize}
\end{frame}

%% page: 307
\begin{frame}
 \frametitle{}
Uteropelvic junction with kidney-
\end{frame}

%% page: 308
\begin{frame}
 \frametitle{}
\begin{itemize}
\item{last to canalize -» most common site of obstruction (hydronephrosis) in fetus.}
\end{itemize}
\end{frame}

%% page: 309
\begin{frame}
 \frametitle{}
Potter's syndrome
\end{frame}

%% page: 310
\begin{frame}
 \frametitle{}
\begin{itemize}
\item{Bilateral renal agenesis —* oligohydramnios —$>$ limb deformities, facial deformities, pulmonarv' hypoplasia. Caused bv malformation of ureteric bud. \\ Babies who can't "Pee" in utero develop Potter's.}
\end{itemize}
\end{frame}

%% page: 311
\begin{frame}
 \frametitle{}
Horseshoe kidney
\end{frame}

%% page: 312
\begin{frame}
 \frametitle{}
\begin{itemize}
\item{Inferior poles of both kidneys fuse. As thev ascend from pelvis during fetal development, horseshoe kidneys get trapped under inferior mesenteric artery and remain low in the abdomen. Kidnev functions normally. Note: horseshoe kidney (partial fusion) vs, cake kidney (complete fusion).}
\end{itemize}
\end{frame}

%% page: 313
\begin{frame}
 \frametitle{}
female
\end{frame}

%% page: 314
\begin{frame}
 \frametitle{}
\begin{itemize}
\item{Default development, Mesonephric duct degenerates and paranmesonephric duct develops.}
\end{itemize}
\end{frame}

%% page: 315
\begin{frame}
 \frametitle{}
Male
\end{frame}

%% page: 316
\begin{frame}
 \frametitle{}
\begin{itemize}
\item{SRY gene on Y chromosome codes for testis-determining factor. Mullerian inhibiting substance secreted by testes (Sertoli cells) suppresses development of paramesonephric ducts. Increase androgens (Leydig cells) —$>$ development of mesonephric ducts.}
\end{itemize}
\end{frame}

%% page: 317
\begin{frame}
 \frametitle{}
Mesonephric (wolffian) duct
\end{frame}

%% page: 318
\begin{frame}
 \frametitle{}
\begin{itemize}
\item{Develops into male internal structures \\ (except prostate)––Seminal vesicles, Epididymis, Ejaculatory duct, and Ductus \\ deferens. SEED.}
\end{itemize}
\end{frame}

%% page: 319
\begin{frame}
 \frametitle{}
Paramesonephric (müllerian) duct
\end{frame}

%% page: 320
\begin{frame}
 \frametitle{}
\begin{itemize}
\item{Develops into fallopian tube, uterus, and up 1/3 of vagina.}
\end{itemize}
\end{frame}

%% page: 321
\begin{frame}
 \frametitle{}
Bicornuate uterus
\end{frame}

%% page: 322
\begin{frame}
 \frametitle{}
\begin{itemize}
\item{Results from incomplete fusion of the paramesonephric ducts. Associated with urinary tract abnormalities and infertility.}
\end{itemize}
\end{frame}

%% page: 323
\begin{frame}
 \frametitle{}
Genital tubercle
\end{frame}

%% page: 324
\begin{frame}
 \frametitle{}
\begin{itemize}
\item{Glans penis \\ Glans clitoris}
\end{itemize}
\end{frame}

%% page: 325
\begin{frame}
 \frametitle{}
Urogenital sinus
\end{frame}

%% page: 326
\begin{frame}
 \frametitle{}
\begin{itemize}
\item{Corpus spongiosum \\ Vestibular bulbs}
\end{itemize}
\end{frame}

%% page: 327
\begin{frame}
 \frametitle{}
Urogenital sinus
\end{frame}

%% page: 328
\begin{frame}
 \frametitle{}
\begin{itemize}
\item{Bulbourethral glands (of Cowper) \\ Greater vestibular glands(of Bartholin)}
\end{itemize}
\end{frame}

%% page: 329
\begin{frame}
 \frametitle{}
Urogenital sinus
\end{frame}

%% page: 330
\begin{frame}
 \frametitle{}
\begin{itemize}
\item{Prostate gland \\ Urethral and paraurethral glands (of Skene)}
\end{itemize}
\end{frame}

%% page: 331
\begin{frame}
 \frametitle{}
Urogenital folds
\end{frame}

%% page: 332
\begin{frame}
 \frametitle{}
\begin{itemize}
\item{Ventral shaft of penis (penile urethra) \\ Labia minora}
\end{itemize}
\end{frame}

%% page: 333
\begin{frame}
 \frametitle{}
Labioscrotal swelling
\end{frame}

%% page: 334
\begin{frame}
 \frametitle{}
\begin{itemize}
\item{Scrotum \\ Labia majora}
\end{itemize}
\end{frame}

%% page: 335
\begin{frame}
 \frametitle{}
Hypospadias
\end{frame}

%% page: 336
\begin{frame}
 \frametitle{}
\begin{itemize}
\item{abnormal opening of penile urethra on inferior (ventral) side of penis due to failurc of urethral folds to close. \\ Hypospadias is more common than epispadias. Fix hypospadias to prevent UTIs. \\ Hypo is below.}
\end{itemize}
\end{frame}


\end{document}

